% Szglab4
% ===========================================================================
%
\chapter{Részletes tervek}

\thispagestyle{fancy}

\section{Osztályok és metódusok tervei}

%todo ezt szépíteni
\lstset{
	breakatwhitespace=false,
	breaklines=true,
	tabsize=2,
	frame=L,
	numbers=left,
	basicstyle=\small\ttfamily
}

\section{Osztályok leírása}
\subsection{BareHands}
\begin{itemize}
	\item A játékos így ás, ha nincs ásója. A kiválasztott cellán csökkennie kell a hó mennyiségnek ásáskor.
	\item Interfészek:
	\begin{itemize}
		\item DigStrategy
	\end{itemize}
	\item Metódusok:
	\begin{itemize}
		\item bool Dig(Tile t): Csökkenti a tile-on található hó mennyiségét. Minden alkalommal fárasztó az ásás, ezért a visszatérési érték mindig true.
	\end{itemize}
\end{itemize}

\subsection{BareIce}
\begin{itemize}
	\item Ilyen a jégtábla, ha nincs rajta iglu. A jégtáblán nincs védelem a vihar elől.
	\item Ősosztályok:
	\begin{itemize}
		\item Shelter
	\end{itemize}
	\item Metódusok:
	\begin{itemize}
		\item void ChillStorm(Tile t): A paraméterként kapott t Tilen álló játékosok testhője csökken.
		\item void BearAttack(Tile t):
		\item void Break(Tile t):
	\end{itemize}
\end{itemize}

\subsection{BuildStrategy}
\begin{itemize}
	\item A játékos így képes építeni. Iglut vagy sátrat.
	\item Attribútumok:
	\begin{itemize}
		\item count: int:
	\end{itemize}
	\item Metódusok:
	\begin{itemize}
		\item void Build(): Épít valamit a táblára.
		\item void Gain(): Kap egy sátrat.
	\end{itemize}
\end{itemize}

\subsection{BreakingShovel}
\begin{itemize}	
	\item Törhető ásó osztály.
	\item Interfészek:
	\begin{itemize}
		\item Item
	\end{itemize}
	\item Metódusok:
	\begin{itemize}
		\item void GiveTo(Player p): A játékos így kap ásót.
	\end{itemize}
\end{itemize}

\subsection{BreakingShovelDig}
\begin{itemize}
	\item A játékos így ás, ha törhető ásó van nála.
	\item Intefészek:
	\begin{itemize}
		\item DigStrategy
	\end{itemize}
	\item Attribútumok:
	\begin{itemize}
		\item lastUsed: bool: Volt-e használva a körben.
		\item durability: int: Mennyiszer lehet még ásni vele.
	\end{itemize}
	\item Metódusok:
	\begin{itemize}
		\item bool Dig(Tile t): Csökkenti a tile-on található hó mennyiségét.
	\end{itemize}
\end{itemize}

\subsection{CantRescue}
\begin{itemize}
	\item A játékos nem tudja kihúzni a csapattársát. A játékos ilyen állapotban van, ha nincs nála kötél.
	\item Interfészek:
	\begin{itemize}
		\item RescueStrategy
	\end{itemize}
	\item Metódusok:
	\begin{itemize}
		\item void Rescue(Tile water, Tile land): Mivel a játékos ebben az állapotban nem tudja megmenteni a csapattársát, ez a fv nem csinál vele semmit. 
	\end{itemize}
\end{itemize}

\subsection{ChillWaterStrategy}
\begin{itemize}
	\item A jégtábla így hűti a vízbe esett játékosokat. Vízben tartózkodás esetén a játékos testhője csökken, a megvalósított stratégia alapján.
	\item Metódusok:
	\begin{itemize}
		\item abstract void Chill(Tile t): A startégiát megvalósító elem dolga implementálni mi történik.
	\end{itemize}
\end{itemize}

\subsection{DigStrategy}
\begin{itemize}
	\item A játékos így ás.	Ásáskor a cellán a hómennyiség csökken.
	\item Metódusok:
	\begin{itemize}
		\item abstract bool Dig(Tile t): A stratégiát megvalósító elem dolga implementálni mi történik ásáskor. Visszaadja, hogy az ásás fárasztó-e.
	\end{itemize}
\end{itemize}

\subsection{DryLand}
\begin{itemize}
	\item A szárazföld nem hűti a játékosokat. A játékos nincsen vízben.
	\item Interfészek:
	\begin{itemize}
		\item ChillWaterStrategy
	\end{itemize}
	\item Metódusok:
	\begin{itemize}
		\item void Chill(Tile t): A stratégia megvalósítása miatt kér be egy t Tile paramétert, a rajta levő játékossal viszont nem csinál semmit, mert az nincs vízben, nem csökkenti testhőjét.
	\end{itemize}
\end{itemize}

\subsection{Empty}
\begin{itemize}
	\item Nincs jégbe fagyott tárgy. Ez az üres eszköz típus, nem képes semmi extra tulajdonságot biztosítani a tulajdonosnak.
	\item Interfészek:
	\begin{itemize}
		\item Item
	\end{itemize}
	\item Metódusok
	\begin{itemize}
		\item void GiveTo(Player p): A paraméterként kapott játékost nem ruházza fel extra tulajdonsággal, mivel épp nincs itt jégbe fagyott tárgy.
	\end{itemize}
\end{itemize}

\subsection{Entity}
\begin{itemize}
	\item Entitás osztály ami a pályát tartózkodhat.
	\item Metódusok:
	\begin{itemize}
		\item Step(Direction d): Lép valamilyen irányba.
		\item void PlaceOn(Tile t): Ráteszi az entitást egy másik táblára. A kötél használatakor használatos.
		\item void Chill(): Hűti az entitést. A testhője csökken.
		\item ResistWater(): Így viselkedik vízben.
		\item BearAttack(): Így viselkedik, ha megtámadja a medve.
	\end{itemize}
\end{itemize}

\subsection{Eskimo}
\begin{itemize}
	\item Játékos fajta. 5 egységnyi testhővel kezd. Képes iglut építeni. A játékos irányítja.
	\item Ősosztályok:
	\begin{itemize}
		\item Player
	\end{itemize}
	\item Metódusok:
	\begin{itemize}
		\item void BuildIgloo(): Épít egy iglut a mezőre, amin áll. Az iglu megvéd majd a hóvihartól. Beállítja a mező hóvihar stratégiáját Iglusra.
	\end{itemize}
\end{itemize}

\subsection{Food}
\begin{itemize}
	\item Élelem, amit a játékos meg tud enni, hogy növelje a testhőjét. Élelem a pályán lesz található.
	\item Interfészek:
	\begin{itemize}
		\item Item
	\end{itemize}
	\item Metódusok:
	\begin{itemize}
		\item void GiveTo(Player p): A paraméterként kapott játékos kap egy élelmet, az bekerül az élelemtárolójába.
	\end{itemize}
\end{itemize}

\subsection{FoodStore}
\begin{itemize}
	\item A játékos ebben a zsebben tárolja az élelmet.
	\item Attribútumok:
	\begin{itemize}
		\item count: int: Hány élelem van a játékosnál.
	\end{itemize}
	\item Metódusok:
	\begin{itemize}
		\item void feed(Player p): Játékos testhője megnő, az élelem mennyisége csökken, mivel a játékos megeszi azt.
		\item void DecrementCount(): Csökkenti a benne található elemek számát.
		\item void Gain(): növeli a benne található elemek számát.
	\end{itemize}
\end{itemize}

\subsection{Game}
\begin{itemize}
	\item Interface a Model és a Controller között. A játékmesterhez tartozó működést valósítja meg. Felelős a játékban lévő objektumok tárolásáért és létrehozásáért.
	\item Attribútumok:
	\begin{itemize}
		\item players: Player[3..*]: Tárolja a játékosokat.
		\item icefield: Tile[1..*]: Tárolja a pályát alkotó elemeket.
		\item bears: PolarBear[*]: Tárolja a medvé(ke)t, ha több van akkor is.
	\end{itemize}
	\item Metódusok:
	\begin{itemize}
		\item void AddTile(t: Tile): Hozzáad egy cellát a játékhoz.
		\item void AddPlayer(pl: Player): Hozzáad egy játékost a játékhoz.
		\item Tile CreateIce(): Létrehoz egy jégtáblát. Ez a metódus az init szekvencia része.
		\item Tile CreateUnstableIce(): Létrehoz egy instabil jégtáblát. Ez a metódus az init szekvencia része.
		\item Tile CreateSea(): Létrehoz egy vizet. Ez a metódus az init szekvencia része.
		\item Tile CreateHole(): Létrehoz egy lyukat: olyan vizet amit hó fed. Ez a metódus az init szekvencia része.
		\item Player CreateEskimo(): Létrehoz egy eszkimó játékost. Ez a metódus az init szekvencia része.
		\item PolarBear CreatePolarBear(): Létrehoz egy medvét. Ez a metódus az init szekvencia része.
		\item Player CreatePolarExplorer(): Létrehoz egy sarkkutató játékost. Ez a metódus az init szekvencia része.
		\item void GameOver(): Ha vége a játéknak, szól a Controllernek, hogy vesztettünk. Külső metódus.
		\item void Turn(): Ezt a metódust a Controller hívja körönként, a körök vezénylésére szolgál. 
		\item void Victory(): Ha vége a játéknak, szól a Controllernek, hogy nyertünk. Külső metódus.
	\end{itemize}
\end{itemize}

\subsection{Igloo}
\begin{itemize}
	\item Ezen a jégtáblán iglu áll, a játékosok védve vannak a vihartól. Az ilyen táblán nem csökken a viharban a rajta állók testhője.
	\item Ősosztályok:
	\begin{itemize} 
		\item Shelter
	\end{itemize}
	\item Metódusok:
	\begin{itemize}
		\item void ChillStorm(Tile t): A paraméterként kapott cellán álló játákosok testhője nem csökken, mivel igluban vannak.
		\item void BearAttack(Tile t): Így viselkedik a mező ha valaki igluban van és megtámadja a medve.
	\end{itemize}
\end{itemize}

\subsection{Item}
\begin{itemize}
	\item Tárgy, a játékos képes ilyeneket felvenni a cellákról. A tárgyak képesek a játékosak képességeket adni. A tárgyak alapvetően jégbe fagyva vannak a pályán.
	\item Metódusok:
	\begin{itemize}
		\item void GiveTo(p: Player): A jétékos kap valamilyen tárgyat, az Item interfészt megvalóító tárgyak felüldefiniálják ezt.
	\end{itemize}
\end{itemize}

\subsection{Naked}
\begin{itemize}
	\item A játékos védtelen a hideg vízzel szemben. A játékos ha így esik vízbe és nem menekítik ki megfullad.
	\item Interfészek:
	\begin{itemize}
		\item WaterResistanceStrategy
	\end{itemize}
	\item Metódusok:
	\begin{itemize}
		\item void Chill(Player p): Játékosnak nincsen ereje a vízben úszni búvárruha nélkül.
	\end{itemize}
\end{itemize}

\subsection{Part}
\begin{itemize}
	\item Jégbefagyott alkatrész. Csak akkor ásható ki, ha nincs rajta hó.
	\item Interfészek:
	\begin{itemize}
		\item Item
	\end{itemize}
	\item Metódusok:
	\begin{itemize}
		\item void GiveTo(Player p): A játékos tárolójába kerül egy darab a rakétapisztolyból.
	\end{itemize}
\end{itemize}

\subsection{PartStore}
\begin{itemize}
	\item A játékos ebben a zsebben tárolja az alkatrészeket.
	\item Attribútumok:
	\begin{itemize}
		\item count: int: Tárolja hány darab alkatrész van belőle a játékosnál.
	\end{itemize}
	\item Metódusok:
	\begin{itemize}
		\item void Gain(PartStore ps): Átveszi az alkatrészeket a paraméterként kapott alkatrésztárolóból.
		\item void Gain(int n): Megnő az alkatrészek száma, ami a játékosnál van.
	\end{itemize}
\end{itemize}

\subsection{Player}
\begin{itemize}
	\item Játékos osztály, amit a felhasználó irányít a grafikus felületen keresztül. Ilyen típussal nem lehet játszani, csak a leszármazottakkal. Felelsőssége a játékos által a controlleren keresztül kiadott műveletek elvégzése. Tárolja a játékos jelenlegi állapotát.
	\item Ősosztályok:
	\begin{itemize}
		\item Entity
	\end{itemize}
	\item Attribútumok:
	\begin{itemize}
		\item bodyTemp: int: Jelzi a játékos jelenlegi hőmérsékletét, ha 0 akkor megfagy $\rightarrow$ játék vége.
		\item currentTile: Tile: A játékos ismeri a mezőt amin éppen áll.
		\item inventory: Item[*]: Tárolja a játékos tárgyait, amik képességekkel tudjak felruházni őt.
		\item digStrategy: DigStrategy: Eldönti hogyan képes ásni a játékos.
		\item energy: int: Számlálja mennyit mozogott az adott körben a játékos.
		\item foodStore: FoodStore: Tárolja a játékos ételeit.
		\item game: Game: A játékos ismeri a játékot.
		\item partStore: PartStore: Tárolja a játékos rakéta alkatrészeit.
		\item rescueStrategy: RescueStrategy: Eldönti, hogy megmenthet egy játékos egy másikat a vízbeesés után.
		\item waterResistanceStrategy: WaterResistanceStrategy: Eldönti, hogy a játékos hogyan viselkedik vízbeesés esetén.
	\end{itemize}
	\item Metódusok:
	\begin{itemize}
		\item void AssembleFlare(): Összerakja a játék végéhez szükséges rakéta pisztolyt. 1 munkaegység
		\item void Chill(): A testhő 1-el csökken, ha 0 alá megy $\rightarrow$ GameOver.
		\item void DecrementEnergy(): Az energiát csökkentő helper metódus.
		\item void Dig(): Ezt a metódust a Controller hívja. A játékos havat ás. 1 munkaegység
		\item void EatFood(): Ezt a metódust a Controller hívja. A játékos eszik. A testhője megnő 1-el.
		\item void PickUp(): Ezt a metódust a Controller hívja. A játékos felvesz egy tárgyat. 1 munkaegység
		\item void Equip(inventorySlot: int): Ezt a metódust a Controller hívja. A játékos kiválaszt egy tárgyat használatra.
		\item void PlaceOn(Tile t): Init szekvencia része. RopeRescue szekvencia része. Rárak egy játékost egy másik Tile-ra.
		\item void RescueTeammate(direction: int): Ezt a metódust a Controller hívja. A játékos kiment egy másikat a vízből. 1 munkaegység
		\item void ResistWater(): A játékos testhője a WaterResistance szerint változik.
		\item void Step(direction: int): Ezt a metódust a Controller hívja. A játékos lép, ha van még hozzá elég energiája. 1 munkaegység
		\item void ToFoodStore(): Élelem megtalálásához helper metódus.
	\end{itemize}
\end{itemize}

\subsection{PolarBear}
\begin{itemize}
	\item Jegesmedve osztály. Random lépeget a táblán és ha playert talál megtámadja azt.
	\item Ősosztályok:
	\begin{itemize}
		\item Entity
	\end{itemize}
	\item Metódusok:
	\begin{itemize}
		\item Step(Directon d): Lép az adott irányba.
	\end{itemize}
\end{itemize}

\subsection{PolarExplorer}
\begin{itemize}
	\item Játékos fajta. 4 egységnyi testhővel kezd. Képes megnézni egy cella teherbíró képességét. A játékos irányítja.
	\item Ősosztályok:
	\begin{itemize}
		\item Player
	\end{itemize}
	\item Metódusok:
	\begin{itemize}
		\item int Examine(direction: int): A játékos megnézheti, hogy egy adott irányban lévő Tile-nak mennyi a teherbírása.
	\end{itemize}
\end{itemize}

\subsection{RescueStrategy}
\begin{itemize}
	\item A játékos így húzza ki csapattársát a vízből. A játékos így képes megmenteni a vízbe esett csapattársát a szomszédos celláról, a megvalósított stratégia alapján. Kötél szükséges a másik játékos megmentéséhez.
	\item Metódusok:
	\begin{itemize}
		\item abstract void Rescue(Tile water, Tile land): A stratégiát megvalósító elem dolga implementálni mi történik.
	\end{itemize}
\end{itemize}

\subsection{Rope}
\begin{itemize}
	\item Jégbe fagyott kötél. Ezzel lehet megmenteni a vízbe esett csapattársat a szomszédos celláról.
	\item Interfészek:
	\begin{itemize}
		\item Item
	\end{itemize}
	\item Metódusok
	\begin{itemize}
		\item void GiveTo(Player p): A játékos kap egy kötelet. Az bekerül az inventoryjába és a megfelelő stratégiájához is a kötél által adott képesség.
	\end{itemize}
\end{itemize}

\subsection{RopeRescue}
\begin{itemize}
	\item A játékos kihúzza csapattársát a vízből. A játékos így menti meg a szomszédos cellán vízbe esett csapattársát.
	\item Interfészek:
	\begin{itemize}
		\item RescueStrategy
	\end{itemize}
	\item Metódusok:
	\begin{itemize}
		\item void Rescue(Tile water, Tile land): A vízben lévők közül egyvalaki rákerül a kihúzó játékos cellájára.
	\end{itemize}
\end{itemize}

\subsection{ScubaGear}
\begin{itemize}
	\item Jégbe fagyott búvárruha. Ezzel lehet életben maradni a vízben.
	\item Interfészek:
	\begin{itemize}
		\item Item
	\end{itemize}
	\item Metódusok:
	\begin{itemize}
		\item void GiveTo(): A játékos búvárruhát kap. Az bekerül az inventoryjába és a megfelelő stratégiája helyére is a búvárruha által adott képesség.
	\end{itemize}
\end{itemize}

\subsection{ScubaWearing}
\begin{itemize}
	\item A játékos testhője nem csökken a vízben. A játékos nem hal bele, ha a vízben marad.
	\item Interfészek:
	\begin{itemize}
		\item WaterResistanceStrategy
	\end{itemize}
	\item Metódusok:
	\begin{itemize}
		\item void Chill(p: Player): A játékost nem hűti a víz, mivel búvárruhát visel.
	\end{itemize}
\end{itemize}

\subsection{Sea}
\begin{itemize}
	\item Ez a cella tenger, hűti a játékosokat.
	\item Interfészek:
	\begin{itemize}
		\item ChillWaterStrategy
	\end{itemize}
	\item Metódusok:
	\begin{itemize}
		\item void Chill(Tile t): Minden rajta álló testhője csökken a WaterResistanceStrategy szerint.
	\end{itemize}
\end{itemize}

\subsection{Shovel}
\begin{itemize}
	\item Jégbe fagyott ásó. Ezzel lehet több havat eltakarítani a celláról.
	\item Interfészek:
	\begin{itemize}
		\item Item
	\end{itemize}
	\item Metódusok:
	\begin{itemize}
		\item void GiveTo(): A játékos ásót kap, ami bekerül az inventoryjába és a megfelelő stratégiájához is bekerül az ásó által adott képesség.
	\end{itemize}
\end{itemize}

\subsection{ShovelDig}
\begin{itemize}
	\item Egyszer lehet ásni vele fáradság nélkül is.
	\item Interfészek:
	\begin{itemize}
		\item DigStrategy
	\end{itemize}
	\item Attribútumok:
	\begin{itemize}
		\item lastUsed: bool: Volt-e már használva a körben.
	\end{itemize}
	\item Metódusok:
	\begin{itemize}
		\item bool Dig(Tile t): Csökkenti a tile-on található hó mennyiségét. Minden második alkalommal fárasztó.
	\end{itemize}
\end{itemize}

\subsection{Tent}
\begin{itemize}
	\item Sátor osztály. Le lehet rakni táblára.
	\item Ősosztályok:
	\begin{itemize}
		\item Shelter
	\end{itemize}
	\item Metódusok:
	\begin{itemize}
		\item void ChillStorm(Tile t): Így viselkedik a tábla, ha sátor van  rajta hóviharban.
		\item void Break(Tile t): Így viselkedik a tábla, ha eltörik miközben valaki a sátorban van.
	\end{itemize}
\end{itemize}

\subsection{TentKit}
\begin{itemize}
	\item Sátor építését lehetővé teszi.
	\item Interfészek:
	\begin{itemize}
		\item Item
	\end{itemize}
	\item Metódusok:
	\begin{itemize}
		\item void GiveTo(Player p): A játékos így kap sátor alapanyagot.
	\end{itemize}
\end{itemize}

\subsection{Tile}
\begin{itemize}
	\item Cella, ilyenekből áll a jégmező ahol a játékosok játszanak.
	\item Attribútumok:
	\begin{itemize}
		\item chillStormStrategy: ChillStormStrategy: Eldönti, kinek változik a testhője vihar esetén.
		\item chillWaterStrategy: ChillWaterStrategy: Eldönti, kinek változik a testhője víz esetén.
		\item item: Item: Ezt a tárgyat lehet kiásni belőle.
		\item neighborTiles: Tile[*]: Szomszédos cellákat ismer.
		\item occupants: Entity[*]: Rajta lévő entitások.
		\item snow: int: Rajta lévő hómennyiség.
		\item weightLimit: int: Rajta lévő játékosok számának maximuma.		
	\end{itemize}
	\item Metódusok:
	\begin{itemize}
		\item void ChillStorm(): Ezt a metódust a Controller hívja viharban. Hűti a játékosokat, ha nincsenek igluban vagy sátorban.
		\item void ChillWater(): Ezt a metódust a Controller hívja körönként. Hűti a játékosokat, ha ez a cella víz.
		\item void DecrementSnow(): A hómennyiséget csökkentő helper függvény.
		\item Item TakeItem(): A játékos megkapja a tartalmazott tárgyat.
		\item Tile NeighborAt(direction): Visszaadja az adott irányban szomszédos cellát.
		\item StepOn(Player): Játékos rálép a cellára, ha többen vannak mint a korlát, a jégtábla átfordul. A függvény futása során beállítja a megfelelő adattagokat az új értékekre.
		\item StepOff(Player): Játékos lelép a celláról. A függvény futása során beállítja a megfelelő adattagokat az új értékekre.
	\end{itemize}
\end{itemize}

\subsection{WaterResistanceStrategy}
\begin{itemize}
	\item Így reagál a játékos a hideg vízre. A vízben búvárruh nélkül nem lehet mozogni. A vízből ha búvárruha nélkül nem húznak ki, nem lehet életben maradni.
	\item Metódusok:
	\begin{itemize}
		\item abstract void Chill(Player p): A stratégiát megvalósító elem dolga implementálni mi történik.
	\end{itemize}
\end{itemize}

\subsection{Proto}
\begin{itemize}
\item Felelősség\newline
Beolvas parancsokat, értelmezi és futtatja őket.
\item Attribútumok
	\begin{itemize}
		\item \texttt{+game: Game}; \newline
		A teljes játékot tartalmazza.
		\item \texttt{-running: boolean;} \newline
		A parancsok feldolgozása megállítható vele.
		\item \texttt{-parsers: CommandParser[*];} \newline
		Ilyen parancsokat tud feldolgozni.
		\item \texttt{-selectedTile: Tile[0..1];}
		\item \texttt{-selectedPlayer: Player[0..1];}
		\item \texttt{-selectedBear: PolarBear[0..1];}
		\item \texttt{+selectTile(Tile t);} \newline			
		Beállítja a selectedTile-t és lenullozza a selectedPlayert és a selectedBeart.			
		\item \texttt{+selectPlayer(Player t);} \newline
		Beállítja a selectedPlayer-t és lenullozza a selectedTile-t és a selectedBeart.	
		\item \texttt{+selectBear(PolarBear t);} \newline			
		Beállítja a selectedBeart és lenullozza a selectedTile-t és a selectedPlayert.			
		\item \texttt{+hasSelectedTile(): boolean;}
		\item \texttt{+hasSelectedPlayer(): boolean;}
		\item \texttt{+hasSelectedBear(): boolean;}
		\item \texttt{+getSelectedTile(): Tile;}
		Kivételt dob ha nincs kiválaszvta dolog.			
		\item \texttt{+getSelectedPlayer(): Player;} \newline
		Kivételt dob ha nincs kiválaszvta dolog.
		\item \texttt{+getSelectedBear(): PolarBear;} \newline
		Kivételt dob ha nincs kiválaszvta dolog.
	\end{itemize}
\item Metódusok
	\begin{itemize}
		\item \texttt{+Proto();}
		\begin{lstlisting}
create game;
create MessagePrinter(this);
game.subscribe(the message printer);
createParsers();
		\end{lstlisting}
		\item \texttt{-createParsers();} \newline			
		Készít egy-egy példányt a beépített CommandParserekből és feltölti velük a parsers kollekciót.
		\item \texttt{+run();} \newline
		Fut a parancsértelmezés.
		\begin{lstlisting}
running = true;
while (runining) {
	getCommand();
	try {
		command.execute(this);
	} catch (an exception that we threw) {
		print a meaningful error message;
	}
}	
		\end{lstlisting}	
		\item \texttt{+stop();} \newline
		Megáll a parancsértelmezés. A running változó false lesz.
		\item \texttt{-getCommand(): Command;} \newline
		Beolvas egy parancsot a standard bemenetről.
		\begin{lstlisting}
while (true) {
	read line;
	strip comments and trailing whitespace;
	tokenize by spaces;
	if (there are tokens) {
		the first token is the keyword;
		find CommandParser by keyword;
		if (not found) print a meaningful error message;
		else return CommandParser.parse(tokens);
	}
}
		\end{lstlisting}	
		\end{itemize}
\end{itemize}

\subsection{MessagePrinter}
\begin{itemize}
\item Felelősség\newline
Kiírja a konzolra a játék eseményeket.
\item Interfészek\newline
GameObserver
\item Attribútumok
	\begin{itemize}
		\item \texttt{-proto: Proto;}
	\end{itemize}
\item Metódusok
\begin{itemize}
		\item \texttt{+MessagePrinter(proto: Proto);}
		\item \texttt{+victory();} \newline
		Győzelem üzenet kiírása, aztán proto.stop().
		\item \texttt{+gameOver();} \newline
		Vereség üzenet kiírása, aztán proto.stop().
		\item \texttt{+explore(Tile);} \newline
		Tile.weightLimit kiírása.		
	\end{itemize}
\end{itemize}

\subsection{Command}
\begin{itemize}
\item Felelősség\newline
Parancs, végrehajtható formában.
\item Metódusok
\begin{itemize}
		\item \texttt{+execute(state: Proto): abstract void;} \newline
		Végrehajtás az adott állapoton.
		\item \texttt{+toString(): abstract String;} \newline
		Így jelenik meg a konzolon.
	\end{itemize}
\end{itemize}

\subsection{CommandParser}
\begin{itemize}
\item Felelősség\newline
Elkészít egy fajta parancsot.
\item Attribútumok
	\begin{itemize}
		\item \texttt{+/keyword: abstract String \string{readOnly\string};} \newline
		A parancs kulcszava.
	\end{itemize}
\item Metódusok
\begin{itemize}
		\item \texttt{+parse(tokens: String[1..*] \string{seq\string}): abstract Command;} \newline
		 Parancs elkészítése tokenekből.
	\end{itemize}
\end{itemize}

\subsection{TileCommand}
\begin{itemize}
\item Felelősség\newline
\item Interfészek\newline
Command
\item Metódusok
\begin{itemize}
		\item \texttt{+toString(): String;}
		\begin{lstlisting}
return "tile " + snow + " " + weightLimit;
		\end{lstlisting}
		\item \texttt{+execute(state: Proto);} \newline
		Készít egy Tile-t Game.createTile használatával, majd kiválasztja proto.selectTile-el.
	\end{itemize}
\end{itemize}
\subsection{TileCommandParser}
\begin{itemize}
\item Felelősség\newline
\item Interfészek\newline
CommandParser
\item Attribútumok
	\begin{itemize}
		\item \texttt{+/keyword: String = "tile";}
	\end{itemize}
\item Metódusok
\begin{itemize}
		\item \texttt{+parse(tokens: String[1..*] \string{seq\string}): Command;}
		\begin{lstlisting}
snow is the second token as a decimal integer;
if (the thid token equals "*") weightLimit is 999;
else weightLimit is the third token as a decimal integer;
create TileCommand;
		\end{lstlisting}
	\end{itemize}
\end{itemize}

\subsection{BuildingCommand}
\begin{itemize}
\item Felelősség\newline
\item Interfészek\newline
Command
\item Metódusok
\item Attribútumok
	\begin{itemize}
		\item \texttt{-type: String;}
	\end{itemize}
\begin{itemize}
		\item \texttt{+BuildingCommand(type: String);}
		\item \texttt{+toString(): String;}
		\begin{lstlisting}
return "building " + type;
		\end{lstlisting}
		\item \texttt{+execute(state: Proto);}
		\begin{lstlisting}
if (type equals "igloo") create Igloo;
if (type equals "tent") create Tent;
set state.selectedTile.shelter;
		\end{lstlisting}
	\end{itemize}
\end{itemize}
\subsection{BuildingCommandParser}
\begin{itemize}
\item Felelősség\newline
\item Interfészek\newline
CommandParser
\item Attribútumok
	\begin{itemize}
		\item \texttt{+/keyword: String = "building";}
	\end{itemize}
\item Metódusok
\begin{itemize}
		\item \texttt{+parse(tokens: String[1..*] \string{seq\string}): Command;}
		\begin{lstlisting}
the second token is the type;
accept only "igloo" or "tent";
create BuildingCommand;
		\end{lstlisting}
	\end{itemize}
\end{itemize}

\subsection{ItemCommand}
\begin{itemize}
\item Felelősség\newline
\item Interfészek\newline
Command
\item Metódusok
\item Attribútumok
	\begin{itemize}
		\item \texttt{-type: String;}
		\item \texttt{+count: int = 1;}
		\item \texttt{+durability: int = -1;}	
	\end{itemize}
\begin{itemize}
		\item \texttt{+ItemCommand(type: String);}
		\item \texttt{+toString(): String;}
		\begin{lstlisting}
if (count > 1) {
	if (type equals "shovel" and durability > -1)
		return "item shovel " + count + " durability " + durability; 
	else 
		return "item " + type + " " + count;
}
else {
	if (type equals "shovel" and durability > -1)
		return "item shovel durability " + durability; 
	else 
		return "item " + type;
}
		\end{lstlisting}
		\item \texttt{+execute(state: Proto);}
		\begin{lstlisting}
if (state has tile selected and count > 1)
	throw an exception;
if (state has no tile selected and state has no player selected)
	throw an exception;
for (count times) {
	if (type equal "empty") create Emty;
	if (type equal "food") create Food;
	if (type equal "part") create Part;
	if (type equal "scubagear") create ScubaGear;
	if (type equal "rope") create Rope;
	if (type equal "tentkit") create TentKit;
	if (type equal "shovel") {
		if (durability > -1) create BreakingShovel with durability;
		else create Shovel;
	}
	if (state has tile selected)
		set state.selectedTile.item;
	if (state has player selected)
		add item to player inventory;
}	
		\end{lstlisting}
	\end{itemize}
\end{itemize}
\subsection{ItemCommandParser}
\begin{itemize}
\item Felelősség\newline
\item Interfészek\newline
CommandParser
\item Attribútumok
	\begin{itemize}
		\item \texttt{+/keyword: String = "item";}
	\end{itemize}
\item Metódusok
\begin{itemize}
		\item \texttt{+parse(tokens: String[1..*] \string{seq\string}): Command;}
		\begin{lstlisting}
the second token is the type;
accept only "empty", "food", "part", "scubagear", "rope", "tentkit", "shovel"
create ItemCommand with type;
if (type equals "shovel") {
	if (the third token equals "durability") {
		the fourth token is the durability as a decimal integer;
		set the ItemCommand.durability;
	}
	else {
		the third token is the count as a decimal integer;
		set the ItemCommand.count;
		if (the fourth token equals "durability") {
			the fifth token is the durability as a decimal integer;
			set the ItemCommand.durability;
		}
	}
}
else {
	the third token is the count as a decimal integer;
	set the ItemCommand.count;
}
return the ItemCommand;
		\end{lstlisting}
	\end{itemize}
\end{itemize}

\subsection{EquipCommand}
\begin{itemize}
\item Felelősség\newline
\item Interfészek\newline
Command
\item Metódusok
\item Attribútumok
	\begin{itemize}
		\item \texttt{-index: int;}
	\end{itemize}
\begin{itemize}
		\item \texttt{+EquipCommand(index: int);}
		\item \texttt{+EquipCommand();} \newline
		"equip all" parancs. Az index -1;
		\item \texttt{+toString(): String;}
		\begin{lstlisting}
if (index > -1) return "equip " + index;
else return "equip all";
		\end{lstlisting}
		\item \texttt{+execute(state: Proto);}
		\begin{lstlisting}
if (index > -1) 
	state.selectedPlayer.equip(index);
else {
	for (all inventory indices)
		state.selectedPlayer.equip(index);
}
		\end{lstlisting}
	\end{itemize}
\end{itemize}
\subsection{EquipCommandParser}
\begin{itemize}
\item Felelősség\newline
\item Interfészek\newline
CommandParser
\item Attribútumok
	\begin{itemize}
		\item \texttt{+/keyword: String = "equip";}
	\end{itemize}
\item Metódusok
\begin{itemize}
		\item \texttt{+parse(tokens: String[1..*] \string{seq\string}): Command;}
		\begin{lstlisting}
if(the second token equals "all") create EquipCommand;
else {
	the second token is the index as a decimal integer;
	create EquipCommand with index;
}
		\end{lstlisting}
	\end{itemize}
\end{itemize}

\subsection{SelectCommand}
\begin{itemize}
\item Felelősség\newline
\item Interfészek\newline
Command
\item Metódusok
\item Attribútumok
	\begin{itemize}
		\item \texttt{-type: String;}
		\item \texttt{-index: int;}
	\end{itemize}
\begin{itemize}
		\item \texttt{+SelectCommand(type: String, index: int);}
		\item \texttt{+toString(): String;}
		\begin{lstlisting}
if (index > -1) return "equip " + index;
else return "equip all";
		\end{lstlisting}
		\item \texttt{+execute(state: Proto);}
		\begin{lstlisting}
if (type equals "tile") state.selectTile(game.tiles[index]);
if (type equals "polarbear") state.selectBear(game.bears[index]);
if (type equals "player") state.selectPlayer(game.player[index]);
		\end{lstlisting}
	\end{itemize}
\end{itemize}
\subsection{SelectCommandParser}
\begin{itemize}
\item Felelősség\newline
\item Interfészek\newline
CommandParser
\item Attribútumok
	\begin{itemize}
		\item \texttt{+/keyword: String = "select";}
	\end{itemize}
\item Metódusok
\begin{itemize}
		\item \texttt{+parse(tokens: String[1..*] \string{seq\string}): Command;}
		\begin{lstlisting}
the second token is the type;
accept only "tile", "polarbear", "player";
if (the type equals "polarbear" and there is no third token)
	the index is 0;
else the index is the third token as a decimal integer;
create SelectCommand with type and index;
		\end{lstlisting}
	\end{itemize}
\end{itemize}

\subsection{EntityCommand}
\begin{itemize}
\item Felelősség\newline
\item Interfészek\newline
Command
\item Attribútumok
	\begin{itemize}
	\item \texttt{-type: String;}
	\item \texttt{-playerBodyHeat: int;}
	\item \texttt{-playerEnergy: int;}	
	\end{itemize}
\item Metódusok
\begin{itemize}
		\item \texttt{+EntityCommand(-type: String);}
		\item \texttt{+EntityCommand(-type: String, -int: playerBodyHeat);}
		\item \texttt{+EntityCommand(-type: String, -int: playerBodyHeat, -int: playerEnergy);} 
		\item \texttt{+toString(): String;}
		\begin{lstlisting}
if (type equals "eskimo" or "polarexplorer") {
	if (playerBodyHeat > -1){
		if (playerEnergy > -1)
			return "entity " + type + " " + playerBodyHeat + " " + playerEnergy;					
		else
			return "entity " + type + " " + playerBodyHeat;
	}
	else return "entity " + type;			
}
else return "entity polarbear";
		\end{lstlisting}
		\item \texttt{+execute(state: Proto);}
		\begin{lstlisting}
if (type equals "eskimo" or "polarexplorer") {
	if (type equals "eskimo") 
		state.game.createEskimo();
	if (type equals "polarexplorer") 
		state.game.createPolarExplorer();
	if (playerBodyHeat > -1)
		set player bodyHeat;
	if (playerEnergy > -1)
		set player energy;			
	state.selectPlayer();
}
if (type equals "polarbear") {
	state.game.createBear();
	state.selectBear();
}
		\end{lstlisting}
	\end{itemize}
\end{itemize}
\subsection{EntityCommandParser}
\begin{itemize}
\item Felelősség\newline
\item Interfészek\newline
CommandParser
\item Attribútumok
	\begin{itemize}
		\item \texttt{+/keyword: String = "entity";}
	\end{itemize}
\item Metódusok
\begin{itemize}
		\item \texttt{+parse(tokens: String[1..*] \string{seq\string}): Command;}
		\begin{lstlisting}
the second token is the type;
accept only "eskimo", "polarexplorer", "polarbear";
if (there is a third token)
	it is the playerBodyHeat as a decimal integer;
if (there is a fourth token)
	it is the playerEnergy as a decimal integer;
create EntityCommand;
		\end{lstlisting}
	\end{itemize}
\end{itemize}

\subsection{ConnectCommand}
\begin{itemize}
\item Felelősség\newline
\item Interfészek\newline
Command
\item Attribútumok
	\begin{itemize}
		\item \texttt{-indices: int[*];}
	\end{itemize}
\item Metódusok
\begin{itemize}
		\item \texttt{+toString(): String;}
		\begin{lstlisting}
"connect " + the indices joinded by spaces;
		\end{lstlisting}
		\item \texttt{+execute(state: Proto);}
		\begin{lstlisting}
for (each index in indices) {
	add state.game.tiles[index] to the state.currentTile.neightbors collection;
}
		\end{lstlisting}
	\end{itemize}
\end{itemize}
\subsection{ConnectCommandParser}
\begin{itemize}
\item Felelősség\newline
\item Interfészek\newline
CommandParser
\item Attribútumok
	\begin{itemize}
		\item \texttt{+/keyword: String = "connect";}
	\end{itemize}
\item Metódusok
\begin{itemize}
		\item \texttt{+parse(tokens: String[1..*] \string{seq\string}): Command;}
		\begin{lstlisting}
all tokens except the first one are indices as decimal integers;
create ConnectCommand;
		\end{lstlisting}
	\end{itemize}
\end{itemize}

\subsection{StepCommand}
\begin{itemize}
\item Felelősség\newline
\item Interfészek\newline
Command
\item Metódusok
\item Attribútumok
	\begin{itemize}
		\item \texttt{-direction: int;}
	\end{itemize}
\begin{itemize}
		\item \texttt{+StepCommand(direction: int);}
		\item \texttt{+toString(): String;}
		\begin{lstlisting}
return "step " + direction;
		\end{lstlisting}
		\item \texttt{+execute(state: Proto);} \newline
		A kiválasztott játékos lép;
	\end{itemize}
\end{itemize}
\subsection{StepCommandParser}
\begin{itemize}
\item Felelősség\newline
\item Interfészek\newline
CommandParser
\item Attribútumok
	\begin{itemize}
		\item \texttt{+/keyword: String = "step";}
	\end{itemize}
\item Metódusok
\begin{itemize}
		\item \texttt{+parse(tokens: String[1..*] \string{seq\string}): Command;}
		\begin{lstlisting}
the second token is the direction as a decimal integer;
create StepCommand with direction;
		\end{lstlisting}
	\end{itemize}
\end{itemize}

\subsection{RescueCommand}
\begin{itemize}
\item Felelősség\newline
\item Interfészek\newline
Command
\item Metódusok
\item Attribútumok
	\begin{itemize}
		\item \texttt{-direction: int;}
	\end{itemize}
\begin{itemize}
		\item \texttt{+RescueCommand(direction: int);}
		\item \texttt{+toString(): String;}
		\begin{lstlisting}
return "rescue " + direction;
		\end{lstlisting}
		\item \texttt{+execute(state: Proto);} \newline
		A kiválasztott játékos kihúzza csapattársát;
	\end{itemize}
\end{itemize}
\subsection{RescueCommandParser}
\begin{itemize}
\item Felelősség\newline
\item Interfészek\newline
CommandParser
\item Attribútumok
	\begin{itemize}
		\item \texttt{+/keyword: String = "rescue";}
	\end{itemize}
\item Metódusok
\begin{itemize}
		\item \texttt{+parse(tokens: String[1..*] \string{seq\string}): Command;}
		\begin{lstlisting}
the second token is the direction as a decimal integer;
create RescueCommand with direction;
		\end{lstlisting}
	\end{itemize}
\end{itemize}

\subsection{ExamineCommand}
\begin{itemize}
\item Felelősség\newline
\item Interfészek\newline
Command
\item Metódusok
\item Attribútumok
	\begin{itemize}
		\item \texttt{-direction: int;}
	\end{itemize}
\begin{itemize}
		\item \texttt{+ExamineCommand(direction: int);}
		\item \texttt{+toString(): String;}
		\begin{lstlisting}
return "examine " + direction;
		\end{lstlisting}
		\item \texttt{+execute(state: Proto);} \newline
		A kiválasztott sarkkutató felderít. Ha nem sarkkutató van kiválasztva, akkor kivételt dob.
	\end{itemize}
\end{itemize}
\subsection{ExamineCommandParser}
\begin{itemize}
\item Felelősség\newline
\item Interfészek\newline
CommandParser
\item Attribútumok
	\begin{itemize}
		\item \texttt{+/keyword: String = "examine";}
	\end{itemize}
\item Metódusok
\begin{itemize}
		\item \texttt{+parse(tokens: String[1..*] \string{seq\string}): Command;}
		\begin{lstlisting}
the second token is the direction as a decimal integer;
create ExamineCommand with direction;
		\end{lstlisting}
	\end{itemize}
\end{itemize}

\subsection{DigCommand}
\begin{itemize}
\item Felelősség\newline
\item Interfészek\newline
Command
\item Metódusok
\begin{itemize}
		\item \texttt{+toString(): String;}
		\begin{lstlisting}
return "dig";
		\end{lstlisting}
		\item \texttt{+execute(state: Proto);} \newline
		A kiválasztott játékos ás;
	\end{itemize}
\end{itemize}
\subsection{DigCommandParser}
\begin{itemize}
\item Felelősség\newline
\item Interfészek\newline
CommandParser
\item Attribútumok
	\begin{itemize}
		\item \texttt{+/keyword: String = "dig";}
	\end{itemize}
\item Metódusok
\begin{itemize}
		\item \texttt{+parse(tokens: String[1..*] \string{seq\string}): Command;} \newline
		Visszaad egy DigCommandot.
	\end{itemize}
\end{itemize}

\subsection{PickUpCommand}
\begin{itemize}
\item Felelősség\newline
\item Interfészek\newline
Command
\item Metódusok
\begin{itemize}
		\item \texttt{+toString(): String;}
		\begin{lstlisting}
return "pickup";
		\end{lstlisting}
		\item \texttt{+execute(state: Proto);} \newline
		A kiválasztott játékos felvesz egy tárgyat.
	\end{itemize}
\end{itemize}
\subsection{PickUpCommandParser}
\begin{itemize}
\item Felelősség\newline
\item Interfészek\newline
CommandParser
\item Attribútumok
	\begin{itemize}
		\item \texttt{+/keyword: String = "pickup";}
	\end{itemize}
\item Metódusok
\begin{itemize}
		\item \texttt{+parse(tokens: String[1..*] \string{seq\string}): Command;} \newline
		Visszaad egy PickUpCommandot.
	\end{itemize}
\end{itemize}

\subsection{BuildCommand}
\begin{itemize}
\item Felelősség\newline
\item Interfészek\newline
Command
\item Metódusok
\begin{itemize}
		\item \texttt{+toString(): String;}
		\begin{lstlisting}
return "build";
		\end{lstlisting}
		\item \texttt{+execute(state: Proto);} \newline
		A kiválasztott játékos épít.
	\end{itemize}
\end{itemize}
\subsection{BuildCommandParser}
\begin{itemize}
\item Felelősség\newline
\item Interfészek\newline
CommandParser
\item Attribútumok
	\begin{itemize}
		\item \texttt{+/keyword: String = "build";}
	\end{itemize}
\item Metódusok
\begin{itemize}
		\item \texttt{+parse(tokens: String[1..*] \string{seq\string}): Command;} \newline
		Visszaad egy BuildCommandot.
	\end{itemize}
\end{itemize}

\subsection{AssembleCommand}
\begin{itemize}
\item Felelősség\newline
\item Interfészek\newline
Command
\item Metódusok
\begin{itemize}
		\item \texttt{+toString(): String;}
		\begin{lstlisting}
return "assemble";
		\end{lstlisting}
		\item \texttt{+execute(state: Proto);} \newline
		A kiválasztott játékos összerakja a rakétát.
	\end{itemize}
\end{itemize}
\subsection{AssembleCommandParser}
\begin{itemize}
\item Felelősség\newline
\item Interfészek\newline
CommandParser
\item Attribútumok
	\begin{itemize}
		\item \texttt{+/keyword: String = "assemble";}
	\end{itemize}
\item Metódusok
\begin{itemize}
		\item \texttt{+parse(tokens: String[1..*] \string{seq\string}): Command;} \newline
		Visszaad egy AssembleCommandot;
	\end{itemize}
\end{itemize}

\subsection{TurnCommand}
\begin{itemize}
\item Felelősség\newline
\item Interfészek\newline
Command
\item Metódusok
\begin{itemize}
		\item \texttt{+toString(): String;}
		\begin{lstlisting}
return "turn";
		\end{lstlisting}
		\item \texttt{+execute(state: Proto);} \newline
		Új kör kezdődik a játékban.
	\end{itemize}
\end{itemize}
\subsection{TurnCommandParser}
\begin{itemize}
\item Felelősség\newline
\item Interfészek\newline
CommandParser
\item Attribútumok
	\begin{itemize}
		\item \texttt{+/keyword: String = "turn";}
	\end{itemize}
\item Metódusok
\begin{itemize}
		\item \texttt{+parse(tokens: String[1..*] \string{seq\string}): Command;} \newline
		Visszaad egy TurnCommandot;
	\end{itemize}
\end{itemize}

\subsection{StormCommand}
\begin{itemize}
\item Felelősség\newline
\item Interfészek\newline
Command
\item Metódusok
\begin{itemize}
		\item \texttt{+toString(): String;}
		\begin{lstlisting}
return "storm";
		\end{lstlisting}
		\item \texttt{+execute(state: Proto);} \newline
		\begin{lstlisting}
for (each tile in state.game.tiles)
	tile.chillStorm();
		\end{lstlisting}
	\end{itemize}
\end{itemize}
\subsection{StormCommandParser}
\begin{itemize}
\item Felelősség\newline
\item Interfészek\newline
CommandParser
\item Attribútumok
	\begin{itemize}
		\item \texttt{+/keyword: String = "storm";}
	\end{itemize}
\item Metódusok
\begin{itemize}
		\item \texttt{+parse(tokens: String[1..*] \string{seq\string}): Command;} \newline
		Visszaad egy StormCommandot.
	\end{itemize}
\end{itemize}

\subsection{QueryCommand}
\begin{itemize}
\item Felelősség\newline
\item Interfészek\newline
Command
\item Metódusok
\begin{itemize}
		\item \texttt{+toString();}
		\begin{verbatim}{ return "query"; }\end{verbatim}	 
		\item \texttt{+execute(state: Proto);} \newline
		Parancsok formájában írja ki a játék állapotát.		
		\begin{lstlisting}
for (command: makeCommands(state.game))
	print line command.toString();
		\end{lstlisting}
		\item \texttt{-makeCommands(Game game): Command[*] \string{seq\string};} \newline
		A parancsok listázása.
		\begin{lstlisting}
result is a writable collection;
for (each tile in game.tiles) {
	add makeTileCommand(tile) to result;
	if (tile is not instance of BareIce)	
		add makeBuildingCommand(tile) to result;
	if (item is not instance of Empty)
		add makeItemCommand(item) to result;
	for (each entity in tile.occupants) { 
		add makeEntityCommand(entity) to result;
		if (entity is instance of Player) {
			add makePlayerCommand(player) to result;
			add listPlayerEquippedItems(player) to result;
			add "equip all" command to result;
			for (item: player.inventory)
				add makeItemCommand(item) to result;
		}
	}
}	
for (each tile in game.tiles) {
	add makeSelectTileCommand(tile, game) to result;
	add makeConnectCommand(tile, game) to result;
}
return result
		\end{lstlisting}		
		\item \texttt{-listPlayerEquippedItems(player: Player): ItemCommand[*] \string{seq\string};} \newline
		Megvizsgálja, hogy milyen tárgyak vannak a játékos használatában, és listázza azokat.
		\begin{lstlisting}
result is a writable collection;
if (player.buildStrategy.count > 0)
	add makeItemCommand(TentKit, player.buildStrategy.count) to result;
if (player.foodStore.count > 0)
	add makeItemCommand(Food, player.foodStore.count) to result;
if (player.partStore.count > 0)
	add makeItemCommand(Part, player.partStore.count) to result;
if (player.rescueStrategy is instance of RopeRescue)
	add makeItemCommand(Rope) to result;
if (player.waterResistanceStrategy is instance of ScubaWearing)
	add makeItemCommand(ScubaGear) to result;
if (player.digStrategy is instance of ShovelDig)
	add makeItemCommand(Shovel) to result;
if (player.digStrategy is instance of BreakingShovelDig) {
	make BreakingShovel with durability player.digStrategy.durability;
	add makeItemCommand(the BreakingShovel) to result;
}
return result;
		\end{lstlisting}		
		\item \texttt{-makeTileCommand(tile: Tile): TileCommand;} \newline
		Készít egy TileCommandot tile.snow és tile.weightLimit tulajdonságokal.
		\item \texttt{-makeBuildingCommand(tile: Tile): BuildingCommand;} \newline
		Készít egy BuildingCommandot a tile.shelter alapján.
		\item \texttt{-makeItemCommand(item: Item): ItemCommand;} \newline
		Készít egy ItemCommandot, az item típusa alapján. Ha ez BreakingShovel, akkor a durability-t is beleteszi.
		\item \texttt{-makeItemCommand(item: Item, int count): ItemCommand;} \newline
		Készít egy ItemCommandot, számosság megadásával.
		\item \texttt{-makeEntityCommand(entity: Entity): EntityCommand;} \newline
		Készít egy EntityCommandot. Ha Player, akkor a player.bodyHeat és player.energy is bele kerül.
		\item \texttt{-makeSelectTileCommand(tile: Tile, game: Game): SelectCommand;} \newline
		Készít egy SelectCommandot, a tile game.tiles-beli indexével.
		\item \texttt{-makeConnectCommand(tile: Tile, game: Game): ConnectCommand;} \newline
		Készít egy ConnectCommandot. Megkeresi a tile.neightbors indexeit a game.tiles tömbben és azokat rakja a ConnectCommandba.
	\end{itemize} 
\end{itemize}
\subsection{QueryCommandParser}
\begin{itemize}
\item Felelősség\newline
\item Interfészek\newline
CommandParser
\item Attribútumok
	\begin{itemize}
		\item \texttt{+/keyword: String = "query";}
	\end{itemize}
\item Metódusok
\begin{itemize}
		\item \texttt{+parse(tokens: String[1..*] \string{seq\string}): Command;}
		Visszaad egy QueryCommandot.
	\end{itemize}
\end{itemize}

\subsection{Osztály1}
\begin{itemize}
\item Felelősség\newline
\comment{Mi az osztály felelőssége. Kb 1 bekezdés. Ha szükséges, akkor state-chart is.}
\item Ősosztályok\newline
\comment{Mely osztályokból származik (öröklési hierarchia)\newline
Legősebb osztály $\rightarrow$ Ősosztály2 $\rightarrow$ Ősosztály3...}
\item Interfészek\newline
\comment{Mely interfészeket valósítja meg.}
\item Attribútumok\newline
\comment{Milyen attribútumai vannak}
	\begin{itemize}
		\item attribútum1: attribútum jellemzése: mire való, láthatósága (UML jelöléssel), típusa
		\item attribútum2: attribútum jellemzése: mire való, láthatósága (UML jelöléssel), típusa
	\end{itemize}
\item Metódusok\newline
\comment{Milyen publikus, protected és privát  metódusokkal rendelkezik. Metódusonként precíz leírás, ha szükséges, activity diagram is  a metódusban megvalósítandó algoritmusról.}
	\begin{itemize}
		\item int foo(Osztály3 o1, Osztály4 o2): metódus leírása, láthatósága (UML jelöléssel)
		\item int bar(Osztály5 o1): metódus leírása, láthatósága (UML jelöléssel)
	\end{itemize}
\end{itemize}

\subsection{Osztály2}
\begin{itemize}
\item Felelősség\newline
\comment{Mi az osztály felelőssége. Kb 1 bekezdés. Ha szükséges, akkor state-chart is.}
\item Ősosztályok\newline
\comment{Mely osztályokból származik (öröklési hierarchia)\newline
Legősebb osztály $\rightarrow$ Ősosztály2 $\rightarrow$ Ősosztály3...}
\item Interfészek\newline
\comment{Mely interfészeket valósítja meg.}
\item Attribútumok\newline
\comment{Milyen attribútumai vannak}
	\begin{itemize}
		\item attribútum1: attribútum jellemzése: mire való, láthatósága (UML jelöléssel), típusa
		\item attribútum2: attribútum jellemzése: mire való, láthatósága (UML jelöléssel), típusa
	\end{itemize}
\item Metódusok\newline
\comment{Milyen publikus, protected és privát  metódusokkal rendelkezik. Metódusonként precíz leírás, ha szükséges, activity diagram is  a metódusban megvalósítandó algoritmusról.}
	\begin{itemize}
		\item int foo(Osztály3 o1, Osztály4 o2): metódus leírása, láthatósága (UML jelöléssel)
		\item int bar(Osztály5 o1): metódus leírása, láthatósága (UML jelöléssel)
	\end{itemize}
\end{itemize}

\section{A tesztek részletes tervei, leírásuk a teszt nyelvén}
[A tesztek részletes tervei alatt meg kell adni azokat a bemeneti adatsorozatokat, amelyekkel a program működése ellenőrizhető. Minden bemenő adatsorozathoz definiálni kell, hogy az adatsorozat végrehajtásától a program mely részeinek, funkcióinak ellenőrzését várjuk és konkrétan milyen eredményekre számítunk, ezek az eredmények hogyan vethetők össze a bemenetekkel.]

\subsection{Teszteset1}
\begin{itemize}
\item Leírás\newline
\comment{szöveges leírás, kb. 1-5 mondat.}
\item Ellenőrzött funkcionalitás, várható hibahelyek
\item Bemenet\newline
\comment{a proto bemeneti nyelvén megadva (lásd előző anyag)}
\item Elvárt kimenet\newline
\comment{a proto kimeneti nyelvén megadva (lásd előző anyag)}
\end{itemize}

\subsection{Teszteset2}
\begin{itemize}
\item Leírás\newline
\comment{szöveges leírás, kb. 1-5 mondat.}
\item Ellenőrzött funkcionalitás, várható hibahelyek
\item Bemenet\newline
\comment{a proto bemeneti nyelvén megadva (lásd előző anyag)}
\item Elvárt kimenet\newline
\comment{a proto kimeneti nyelvén megadva (lásd előző anyag)}
\end{itemize}

\section{A tesztelést támogató programok tervei}
\comment{A tesztadatok előállítására, a tesztek eredményeinek kiértékelésére szolgáló segédprogramok részletes terveit kell elkészíteni.}

