% Szglab4
% ===========================================================================
%
\chapter{Prototípus beadása}

\thispagestyle{fancy}

\section{Fordítási és futtatási útmutató}
\comment{A feltöltött program fordításával és futtatásával kapcsolatos útmutatás. Ennek tartalmaznia kell leltárszerűen az egyes fájlok pontos nevét, méretét byte-ban, keletkezési idejét, valamint azt, hogy a fájlban mi került megvalósításra.}

\subsection{Fájllista}

\begin{fajllista}

% Console.Write(string.Join(Environment.NewLine, Directory.EnumerateFiles(@"proto\src", "*.java", SearchOption.AllDirectories).Select(x => new FileInfo(x)).Select(x => $"\\fajl{{{x.Name}}}{{{x.Length / 1000.0:0.00}KB}}{{{x.CreationTime:yyy.MM.dd~HH:mm~}}}{{{Path.GetFileNameWithoutExtension(x.Name)} osztály forráskódja.}}")));

\fajl{AssembleCommand.java}{0,29KB}{2020.04.21~22:28~}{AssembleCommand osztály forráskódja.}
\fajl{AssembleCommandParser.java}{0,31KB}{2020.04.21~22:31~}{AssembleCommandParser osztály forráskódja.}
\fajl{BuildCommand.java}{0,38KB}{2020.04.21~22:27~}{BuildCommand osztály forráskódja.}
\fajl{BuildCommandParser.java}{0,30KB}{2020.04.21~22:32~}{BuildCommandParser osztály forráskódja.}
\fajl{BuildingCommand.java}{0,72KB}{2020.04.21~22:20~}{BuildingCommand osztály forráskódja.}
\fajl{BuildingCommandParser.java}{0,32KB}{2020.04.21~22:32~}{BuildingCommandParser osztály forráskódja.}
\fajl{Command.java}{0,13KB}{2020.04.21~22:12~}{Command osztály forráskódja.}
\fajl{CommandParser.java}{0,17KB}{2020.04.21~22:14~}{CommandParser osztály forráskódja.}
\fajl{ConnectCommand.java}{1,41KB}{2020.04.21~22:22~}{ConnectCommand osztály forráskódja.}
\fajl{ConnectCommandParser.java}{0,60KB}{2020.04.21~22:33~}{ConnectCommandParser osztály forráskódja.}
\fajl{DigCommand.java}{0,27KB}{2020.04.21~22:26~}{DigCommand osztály forráskódja.}
\fajl{DigCommandParser.java}{0,29KB}{2020.04.21~22:33~}{DigCommandParser osztály forráskódja.}
\fajl{EatCommand.java}{0,28KB}{2020.04.21~22:28~}{EatCommand osztály forráskódja.}
\fajl{EatCommandParser.java}{0,29KB}{2020.04.21~22:33~}{EatCommandParser osztály forráskódja.}
\fajl{EntityCommand.java}{1,90KB}{2020.04.21~22:21~}{EntityCommand osztály forráskódja.}
\fajl{EntityCommandParser.java}{1,79KB}{2020.04.21~22:34~}{EntityCommandParser osztály forráskódja.}
\fajl{EquipCommand.java}{1,57KB}{2020.04.21~22:21~}{EquipCommand osztály forráskódja.}
\fajl{EquipCommandParser.java}{1,33KB}{2020.04.21~22:34~}{EquipCommandParser osztály forráskódja.}
\fajl{ExamineCommand.java}{1,11KB}{2020.04.21~22:26~}{ExamineCommand osztály forráskódja.}
\fajl{ExamineCommandParser.java}{1,03KB}{2020.04.21~22:35~}{ExamineCommandParser osztály forráskódja.}
\fajl{ItemCommand.java}{2,40KB}{2020.04.21~22:20~}{ItemCommand osztály forráskódja.}
\fajl{ItemCommandParser.java}{1,85KB}{2020.04.21~22:35~}{ItemCommandParser osztály forráskódja.}
\fajl{Main.java}{0,15KB}{2020.04.21~20:45~}{Main osztály forráskódja.}
\fajl{MessagePrinter.java}{0,89KB}{2020.04.21~22:11~}{MessagePrinter osztály forráskódja.}
\fajl{PickUpCommand.java}{0,28KB}{2020.04.21~22:27~}{PickUpCommand osztály forráskódja.}
\fajl{PickUpCommandParser.java}{0,47KB}{2020.04.21~22:36~}{PickUpCommandParser osztály forráskódja.}
\fajl{Proto.java}{6,13KB}{2020.04.21~22:10~}{Proto osztály forráskódja.}
\fajl{ProtoException.java}{0,38KB}{2020.04.26~11:33~}{ProtoException osztály forráskódja.}
\fajl{QueryCommand.java}{5,62KB}{2020.04.21~22:22~}{QueryCommand osztály forráskódja.}
\fajl{QueryCommandParser.java}{0,30KB}{2020.04.21~22:36~}{QueryCommandParser osztály forráskódja.}
\fajl{RescueCommand.java}{0,59KB}{2020.04.21~22:25~}{RescueCommand osztály forráskódja.}
\fajl{RescueCommandParser.java}{0,69KB}{2020.04.21~22:36~}{RescueCommandParser osztály forráskódja.}
\fajl{SelectCommand.java}{0,98KB}{2020.04.21~22:21~}{SelectCommand osztály forráskódja.}
\fajl{SelectCommandParser.java}{0,85KB}{2020.04.21~22:36~}{SelectCommandParser osztály forráskódja.}
\fajl{StepCommand.java}{0,83KB}{2020.04.21~22:24~}{StepCommand osztály forráskódja.}
\fajl{StepCommandParser.java}{0,68KB}{2020.04.21~22:37~}{StepCommandParser osztály forráskódja.}
\fajl{StormCommand.java}{0,33KB}{2020.04.21~22:28~}{StormCommand osztály forráskódja.}
\fajl{StormCommandParser.java}{0,47KB}{2020.04.21~22:37~}{StormCommandParser osztály forráskódja.}
\fajl{TileCommand.java}{0,58KB}{2020.04.21~22:19~}{TileCommand osztály forráskódja.}
\fajl{TileCommandParser.java}{0,85KB}{2020.04.21~22:37~}{TileCommandParser osztály forráskódja.}
\fajl{TurnCommand.java}{0,24KB}{2020.04.21~22:28~}{TurnCommand osztály forráskódja.}
\fajl{TurnCommandParser.java}{0,47KB}{2020.04.21~22:37~}{TurnCommandParser osztály forráskódja.}
\fajl{BareHands.java}{0,76KB}{2020.04.22~16:27~}{BareHands osztály forráskódja.}
\fajl{BareIce.java}{1,23KB}{2020.04.22~16:27~}{BareIce osztály forráskódja.}
\fajl{BreakingShovel.java}{1,18KB}{2020.04.22~16:27~}{BreakingShovel osztály forráskódja.}
\fajl{BreakingShovelDig.java}{1,80KB}{2020.04.22~16:27~}{BreakingShovelDig osztály forráskódja.}
\fajl{BuildStrategy.java}{0,80KB}{2020.04.22~16:27~}{BuildStrategy osztály forráskódja.}
\fajl{CantRescue.java}{0,58KB}{2020.04.22~16:27~}{CantRescue osztály forráskódja.}
\fajl{ChillWaterStrategy.java}{0,44KB}{2020.04.22~16:27~}{ChillWaterStrategy osztály forráskódja.}
\fajl{DigStrategy.java}{0,48KB}{2020.04.22~16:27~}{DigStrategy osztály forráskódja.}
\fajl{DryLand.java}{0,61KB}{2020.04.22~16:27~}{DryLand osztály forráskódja.}
\fajl{Empty.java}{0,49KB}{2020.04.22~16:27~}{Empty osztály forráskódja.}
\fajl{Entity.java}{1,58KB}{2020.04.22~16:27~}{Entity osztály forráskódja.}
\fajl{Eskimo.java}{0,74KB}{2020.04.22~16:27~}{Eskimo osztály forráskódja.}
\fajl{Food.java}{0,50KB}{2020.04.22~16:27~}{Food osztály forráskódja.}
\fajl{FoodStore.java}{0,81KB}{2020.04.22~16:27~}{FoodStore osztály forráskódja.}
\fajl{Game.java}{5,29KB}{2020.04.22~16:27~}{Game osztály forráskódja.}
\fajl{GameObserver.java}{0,42KB}{2020.04.22~16:27~}{GameObserver osztály forráskódja.}
\fajl{Igloo.java}{0,79KB}{2020.04.22~16:27~}{Igloo osztály forráskódja.}
\fajl{Item.java}{0,47KB}{2020.04.22~16:27~}{Item osztály forráskódja.}
\fajl{Naked.java}{0,41KB}{2020.04.22~16:27~}{Naked osztály forráskódja.}
\fajl{Part.java}{0,31KB}{2020.04.22~16:27~}{Part osztály forráskódja.}
\fajl{PartStore.java}{0,63KB}{2020.04.22~16:27~}{PartStore osztály forráskódja.}
\fajl{Player.java}{6,73KB}{2020.04.22~16:27~}{Player osztály forráskódja.}
\fajl{PolarBear.java}{0,76KB}{2020.04.22~16:27~}{PolarBear osztály forráskódja.}
\fajl{PolarExplorer.java}{0,53KB}{2020.04.22~16:27~}{PolarExplorer osztály forráskódja.}
\fajl{RescueStrategy.java}{0,25KB}{2020.04.22~16:27~}{RescueStrategy osztály forráskódja.}
\fajl{Rope.java}{0,38KB}{2020.04.22~16:27~}{Rope osztály forráskódja.}
\fajl{RopeRescue.java}{0,54KB}{2020.04.22~16:27~}{RopeRescue osztály forráskódja.}
\fajl{ScubaGear.java}{0,41KB}{2020.04.22~16:27~}{ScubaGear osztály forráskódja.}
\fajl{ScubaWearing.java}{0,32KB}{2020.04.22~16:27~}{ScubaWearing osztály forráskódja.}
\fajl{Sea.java}{0,33KB}{2020.04.22~16:27~}{Sea osztály forráskódja.}
\fajl{Shelter.java}{0,88KB}{2020.04.22~16:27~}{Shelter osztály forráskódja.}
\fajl{Shovel.java}{0,54KB}{2020.04.22~16:27~}{Shovel osztály forráskódja.}
\fajl{ShovelDig.java}{0,47KB}{2020.04.22~16:27~}{ShovelDig osztály forráskódja.}
\fajl{Tent.java}{0,66KB}{2020.04.22~16:27~}{Tent osztály forráskódja.}
\fajl{TentKit.java}{0,33KB}{2020.04.22~16:27~}{TentKit osztály forráskódja.}
\fajl{Tile.java}{4,46KB}{2020.04.22~16:27~}{Tile osztály forráskódja.}
\fajl{WaterResistanceStrategy.java}{0,41KB}{2020.04.22~16:27~}{WaterResistanceStrategy osztály forráskódja.}

\end{fajllista}

\subsection{Fordítás}
\comment{A fenti listában szereplő forrásfájlokból milyen műveletekkel lehet a bináris, futtatható kódot előállítani. Az előállításhoz csak a 2. Követelmények c. dokumentumban leírt környezetet szabad előírni.}

\lstset{escapeinside=`', xleftmargin=10pt, frame=single, basicstyle=\ttfamily\footnotesize, language=sh}
\begin{lstlisting}
javac -d bin *.java
\end{lstlisting}

\subsection{Futtatás}
\comment{A futtatható kód elindításával kapcsolatos teendők leírása. Az indításhoz csak a 2. Követelmények c. dokumentumban leírt környezetet szabad előírni.}

\lstset{escapeinside=`', xleftmargin=10pt, frame=single, basicstyle=\ttfamily\footnotesize, language=sh}
\begin{lstlisting}
cd bin
java Main.java
\end{lstlisting}


\section{Tesztek jegyzőkönyvei}

\subsection{Teszteset1}
\comment{Az alábbi táblázatot az utolsó, sikeres tesztfuttatáshoz kell kitölteni}

\tesztok{...}{...}

\comment{Az alábbi táblázatot a megismételt (hibás) tesztek esetén kell kitölteni minden ismétléshez egyszer. Ha szükséges, akkor a valós kimenet is mellékelhető mint a teszt eredménye.}

\tesztfail{...}{...}{...}{...}{...}

\section{Értékelés}
\comment{A projekt kezdete óta az értékelésig eltelt időben tagokra bontva, százalékban.}

\begin{ertekeles}
\tag{Horváth} % Tag neve
{23.5}        % Munka szazalekban
\tag{Német}
{24.5}
\tag{Tóth}
{25}
\tag{Oláh}
{27}
\end{ertekeles}

