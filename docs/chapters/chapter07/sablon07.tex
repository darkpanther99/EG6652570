% Szglab4
% ===========================================================================
%

\chapter{Prototípus koncepciója}

\thispagestyle{fancy}

\section{Prototípus interface-definíciója}
%\comment{Definiálni kell a teszteket leíró nyelvet. Külön figyelmet kell fordítani arra, hogy ha a rendszer véletlen elemeket is tartalmaz, akkor a véletlenszerűség ki-bekapcsolható legyen, és a program determinisztikusan is tesztelhető legyen.}

\subsection{Az interfész általános leírása}
%\comment{A protó (karakteres) input és output felületeit úgy kell kialakítani, hogy az input fájlból is vehető legyen illetőleg az output fájlba menthető legyen, vagyis kommunikációra csak a szabványos be- és kimenet használható.}
A program egysoros parancsokat vár a standard bemeneten. A játék kezdőállapotát definíciós parancsokkal kell megadni, majd vezérő parancsokkal lehet játszani. A \texttt{query} speciális parancs hatására, a játék teljes jelenlegi állapota kiíródik a standard kimenetre, definíciós parancsok sorozataként.

\subsection{Bemeneti nyelv}
%\comment{Definiálni kell a teszteket leíró nyelvet. Külön figyelmet kell fordítani arra, hogy ha a rendszer véletlen elemeket is tartalmaz, akkor a véletlenszerűség ki-bekapcsolható legyen, és a program determinisztikusan is futtatható legyen. A szálkezelést is tesztelhető, irányítható módon kell megoldani.}
Következik, a program által elfogadott bemeneti nyelvtan Extended Backus-Naur formában:

\definecolor{mymaroon}{rgb}{0.5, 0.07, 0}
\definecolor{myblue}{rgb}{0, 0.02, 0.9}
\lstset{
	breakatwhitespace=false,
	keepspaces=false,
	breaklines=true,
	frame=L,
	numbers=left,
	language=C, %kicsit hasonlít a C-re
	basicstyle=\small\ttfamily,
	identifierstyle=\color{myblue},
	stringstyle=\color{mymaroon},
	caption={A bemeneti nyelvtan.},
	label=GrammarEBNF
}
\lstinputlisting{chapters/chapter07/grammar.ebnf}

Szótár:

\begin{itemize}
\item \texttt{action\textunderscore{}command}
A jelenleg kiválasztott játékos cselekvése.
\item \texttt{assemble\textunderscore{}command}
A jelenleg kiválasztott játékos összeszereli a rakétát.
\item \texttt{build\textunderscore{}command}
A jelenleg kiválasztott játékos iglut/sátrat épít.
\item \texttt{building\textunderscore{}command}
A jelenleg kiválasztott cellára iglu/sátor épül.
\item \texttt{building\textunderscore{}type}
Iglu vagy sátor.
\item \texttt{command\textunderscore{}end}
Parancsok közti karaktersorozat, amit nem értelmezünk.
\item \texttt{comment}
\# karaktertől a sor végéig lehet komment.
\item \texttt{dig\textunderscore{}command}
A jelenleg kiválasztott játékos havat lapátol.
\item \texttt{direction}
A pálya négyzetrácson értelmezett irány.
\item \texttt{entity\textunderscore{}command}
Egy entitás létrehozása.
\item \texttt{entity\textunderscore{}definition}
Egy entitás létrehozása és tulajdonságainak beállítása.
\item \texttt{equip\textunderscore{}all\textunderscore{}command}
A jelenleg kiválasztott játékos felveszi az összes birtokában lévő tárgyat.
\item \texttt{equip\textunderscore{}command}
A jelenleg kiválasztott játékos felvesz birtokában lévő tárgyat.
\item \texttt{equip\textunderscore{}index\textunderscore{}command}
A jelenleg kiválasztott játékos felveszi az adott birtokában lévő tárgyat.
\item \texttt{examine\textunderscore{}command}
A jelenleg kiválasztott sarkkutató játékos felderít egy cellát.
\item \texttt{game}
Parancsok helyes sorozata.
\item \texttt{grid\textunderscore{}command}
Meghatározza a pálya négyzetrács méretét.
\item \texttt{grid\textunderscore{}definition}
A pálya létrehozása. grid\textunderscore{}height * grid\textunderscore{}width darab cella sorfolytonos elrendezésben.
\item \texttt{grid\textunderscore{}height}
A pálya négyzetrács magassága.
\item \texttt{grid\textunderscore{}width}
A pálya négyzetrács szélessége.
\item \texttt{integer}
Nemnegatív egész szám, tízes számrendszerben.
\item \texttt{inventory\textunderscore{}index}
Egy - játékos által birtokolt - tárgy száma.
\item \texttt{item\textunderscore{}command}
Tárgy/Tárgyak létrehozása.
\item \texttt{item\textunderscore{}command\textunderscore{}multiple}
Több tárgy létrehozása.
\item \texttt{item\textunderscore{}command\textunderscore{}single}
Egy tárgy létrehozása.
\item \texttt{item\textunderscore{}count}
Tárgyak számának megadása.
\item \texttt{item\textunderscore{}durability}
A törékeny lapát tárgy hátramaradt használásainak száma.
\item \texttt{item\textunderscore{}type}
Tárgy megadása.
\item \texttt{pickup\textunderscore{}command}
A jelenleg kiválasztott játékos kiás egy tárgyat a jégből.
\item \texttt{player\textunderscore{}actions}
Egy játékos cselekedetei.
\item \texttt{player\textunderscore{}bodyheat}
Egy játékos testhője.
\item \texttt{player\textunderscore{}class}
Játékos osztály megadása: eszkimó/sarkkutató.
\item \texttt{player\textunderscore{}command}
Egy játékos létrehozása.
\item \texttt{player\textunderscore{}definition}
Egy játékos létrehozása, és birtokolt tárgyainak beállítása.
\item \texttt{player\textunderscore{}energy}
Játékos hátramaradt energiája.
\item \texttt{player\textunderscore{}equipped\textunderscore{}items}
Ezeket a tárgyakat a játékos felveszi.
\item \texttt{player\textunderscore{}index}
Játékos száma.
\item \texttt{player\textunderscore{}inventory\textunderscore{}items}
Ezek a tárgyak a játékos birtokába kerülnek.
\item \texttt{polarbear\textunderscore{}actions}
A jelenleg kiválasztott jegesmedve cselekedetei.
\item \texttt{polarbear\textunderscore{}command}
A jelenleg kiválasztott jegesmedve létrehozása.
\item \texttt{polarbear\textunderscore{}index}
Jegesmedve száma.
\item \texttt{query\textunderscore{}command}
A parancs hatására \texttt{grid\textunderscore{}definition} íródik ki az stdout-ra, ami reprezentálja a játék jelenlegi állapotát.
\item \texttt{rescue\textunderscore{}command}
A jelenleg kiválasztott játékos kíhúzza csapattársát a vízből.
\item \texttt{select\textunderscore{}command}
Entitás kiválasztása.
\item \texttt{select\textunderscore{}player\textunderscore{}command}
Kiválaszt egy játékost.
\item \texttt{select\textunderscore{}polarbear\textunderscore{}command}
Kiválaszt egy jegesmedvét.
\item \texttt{step\textunderscore{}command}
A jelenleg kiválasztott játékos lép.
\item \texttt{storm\textunderscore{}command}
Hóvihar kezdődik.
\item \texttt{tile\textunderscore{}command}
Egy cella létrehozása.
\item \texttt{tile\textunderscore{}definition}
Egy cella létrehozása, és a rajta lévő dolgok létrehozása.
\item \texttt{tile\textunderscore{}snow}
A cella hószintjének definiálása.
\item \texttt{tile\textunderscore{}weight\textunderscore{}limit}
A cella teherbírásának definiálása: hány entitás állhat rajta. \newline A * karakter végtelen teherbírást jelöl. A 0 teherbírású cella tenger.
\item \texttt{turn\textunderscore{}command}
A parancs hatására új kör kezdődik a játékban.
\item \texttt{turn\textunderscore{}definition}
Egy kör során végrehajtott parancsok sorozata.
\end{itemize}

%\comment{Ha szükséges, meg kell adni a konfigurációs (pl. pályaképet megadó) fájlok nyelvtanát is.}

\subsection{Kimeneti nyelv}
%\comment{Egyértelműen definiálni kell, hogy az egyes bemeneti parancsok végrehajtása után előálló állapot milyen formában jelenik meg a szabványos kimeneten.}
A kimeneti nyelv a bemeneti nyelv részhalmaza. Pontosabban, egy  \texttt{grid\textunderscore{}definition} nyelvi elem. Kivételes esetek a következők:
\begin{itemize}
\item Sarkkutató felderít egy cellát.
	\begin{itemize}
	\item Az \texttt{examine\textunderscore{}command} lefutását követően üzenet jelenik meg a standard kimeneten: \newline \texttt{"Tile weight limit: N\textbackslash{}n"}, ahol N a cella teherbírása.
	\end{itemize}
\item Egy játékos meghal.
	\begin{itemize}
	\item Üzenet jelenik meg: \texttt{"Game over.\textbackslash{}n"}, és a program megáll.
	\end{itemize}
\item A játékosok összeszerelik a rakétát.
	\begin{itemize}
	\item Üzenet jelenik meg: \texttt{"Victory.\textbackslash{}n"}, és a program megáll.
	\end{itemize}
\end{itemize}

\section{Összes részletes use-case}
%\comment{A use-case-eknek a részletezettsége feleljen meg a kezelői felületnek, azaz a felület elemeire kell hivatkozniuk. Alábbi táblázat minden use-case-hez külön-külön.}

\begin{figure}[h]
\begin{center}
%\includegraphics[width=17cm]{chapters/chapter07/usecase.png}
\caption{x}
\label{fig:ProtoUseCase}
\end{center}
\end{figure}

\usecase{grid}{Pályaméret beállítása.}{Proto}{
	1. Jön egy \texttt{grid\textunderscore{}command}.\newline
	2. A pályaméret beállítódik.
}
\usecase{tile}{Cella létrehozása.}{Proto}{
	1. Jön egy \texttt{tile\textunderscore{}command}.\newline
	2. Létrejön az adott cella.
}
\usecase{building}{Épület létrehozása.}{Proto}{
	1. Jön egy \texttt{building\textunderscore{}command}.\newline
	2. Létrejön az adott épület.
}
\usecase{item}{Tárgy létrehozása.}{Proto}{
	1. Jön egy \texttt{item\textunderscore{}command}.\newline
	2. Létrejön az adott tárgy, vagy tárgyak.
}
\usecase{equip}{Tárgy felvétele.}{Proto}{
	1. Jön egy \texttt{equip\textunderscore{}command}.\newline
	2. A jelenleg kiválasztott játékos felvesz egy tárgyat.
}
\usecase{entity}{Entitás létrehozása.}{Proto}{
	1. Jön egy \texttt{entity\textunderscore{}command}.\newline
	2. Létrejön az adott entitás.
}
\usecase{select}{Entitás kiválasztása.}{Proto}{
	1. Jön egy \texttt{select\textunderscore{}command}.\newline
	2. Kiválasztódik az adott entitás
}
\usecase{turn}{Új kör kezdése.}{Proto}{
	1. Jön egy \texttt{turn\textunderscore{}command}.\newline
	2. Új kör kezdődik.
}
\usecase{storm}{Vihar kezdése.}{Proto}{
	1. Jön egy \texttt{storm\textunderscore{}command}.\newline
	2. Vihar kezdődik.
}
\usecase{step}{Entitás lép.}{Proto}{
	1. Jön egy \texttt{step\textunderscore{}command}.\newline
	2. A jelenleg kiválasztott entitás lép.
}
\usecase{rescue}{Játékos megment.}{Proto}{
	1. Jön egy \texttt{rescue\textunderscore{}command}.\newline
	2. A jelenleg kiválasztott játékos kíhúzza csapattársát a vízből.
}
\usecase{dig}{Játékos lapátol.}{Proto}{
	1. Jön egy \texttt{dig\textunderscore{}command}.\newline
	2. A jelenleg kiválasztott játékos lapátol.
}
\usecase{pickup}{Játékos felvesz.}{Proto}{
	1. Jön egy \texttt{pickup\textunderscore{}command}.\newline
	2. A jelenleg kiválasztott játékos kiás egy tárgyat.
}
\usecase{build}{Játékos épít.}{Proto}{
	1. Jön egy \texttt{build\textunderscore{}command}.\newline
	2. A jelenleg kiválasztott játékos épít.
}
\usecase{assemble}{Játékos összeszerel}{Proto}{
	1. Jön egy \texttt{assemble\textunderscore{}command}.\newline
	2. A jelenleg kiválasztott játékos összeszereli a rakétát.
}
\usecase{examine}{Jégtábla felderítése}{Proto}{
	1. Jön egy \texttt{examine\textunderscore{}command}.\newline
	2. A jelenleg kiválasztott sarkkutató játékos felderít egy mezőt.
}
\usecase{query}{A játékállapot lekérdezése.}{Proto}{
	1. Jön egy \texttt{query\textunderscore{}command}.\newline
	2. A játék állapota kiíródik a standard kimenetre.
}

\section{Tesztelési terv}
%\comment{A tesztelési tervben definiálni kell, hogy a be- és kimeneti fájlok egybevetésével miként végezhető el a program tesztelése. Meg kell adni teszt forgatókönyveket. Az egyes teszteket elég informálisan, szabad szövegként leírni. Teszt-esetenként egy-öt mondatban. Minden teszthez meg kell adni, hogy mi a célja, a proto mely funkcionalitását, osztályait stb. teszteli. Az alábbi táblázat minden teszt-esethez külön-külön elkészítendő.}

\teszteset{PickUpFood}{A játékos az adott mezőre lép. A mezőn élelem található. A játékos felveszi és a saját zsebébe rakja az ételt.}{Item felvétel teszt.}
\teszteset{PickUpPart}{A játékos adott mezőre lép. A mezőn rakéta pisztoly darab található. A játékos felveszi és a megfelelő zsebébe rakja.}{Item felvétel teszt.}
\teszteset{PickUpShovel}{A játékos adott mezőre lép. A mezőn ásó található. A játékos felvesz és a zsebébe rakja. A játékos ez utána már tud ásni később.}{Item felvétel teszt.}
\teszteset{PickUpBreakableShovel}{A játékos adott mezőre lép. A mezőn törékeny ásó található. A játékos felvesz és a zsebébe rakja. A játékos ez utána már tud ásni később.}{Item felvétel teszt.}
\teszteset{PickUpRope}{A játékos adott mezőre lép. A mezőn kötél található. A játékos felvesz és a zsebébe rakja. A játékos ez utána már ki tudja menteni a csapattársait.}{Item felvétel teszt.}
\teszteset{PickUpScubaGear}{A játékos adott mezőre lép. A mezőn búvárruha található. A játékos felvesz és a zsebébe rakja. A játékos ez utána már képes túlélni a vízben.}{Item felvétel teszt.}
\teszteset{PickUpTent}{A játékos adott mezőre lép. A mezőn sátor található. A játékos felvesz és a zsebébe rakja. A játékos ez utána már képes sátrat építeni egy mezőre.}{Item felvétel teszt.}
\teszteset{BareHandsDig}{A játékos azon a mezőn ahol éppen áll havat lapátol. A hómennyiség az adott mezőn csökken}{Hó lapátolás.}
\teszteset{ShovelDig}{A játékos azon a mezőn ahol éppen áll havat lapátol, van nála lapát. A hómennyiség az adott mezőn csökken}{Hó lapátolás.}
\teszteset{BreakingShovelDig}{A játékos azon a mezőn ahol éppen áll havat lapátol, van nála törékeny lapát. A hómennyiség az adott mezőn csökken. A lapát még nem törik el.}{Hó lapátolás.}
\teszteset{BreakingShovelDig}{A játékos azon a mezőn ahol éppen áll havat lapátol, van nála törékeny lapát. A hómennyiség az adott mezőn csökken. A lapát eltörik}{Hó lapátolás.}
\teszteset{StepOnStableIce}{A játékos stabil jégre lép.}{Játékos mezőre lép.}
\teszteset{StepOnUnstableIceWithScubaGearBreaking}{A játékos instabil jégre lép. A játékoson van búvárruha. A jég eltörik, mert nem bírja el. Minden játékos vízbe esik aki ott volt.}{Játékos mezőre lép.}
\teszteset{StepOnUnstableIceWithScubaGearCanHold}{A játékos instabil jégre lép. A játékoson van búvárruha. A jég nem törik el.}{Játékos mezőre lép.}
\teszteset{StepOnUnstableIceNakedBreaking}{A játékos instabil jégre lép. A játékoson nincs búvárruha. A jég eltörik, mert nem bírja el. Minden játékos vízbe esik aki ott volt.}{Játékos mezőre lép.}
\teszteset{StepOnUnstableIceNakedCanHold}{A játékos instabil jégre lép. A játékoson nincs búvárruha. A jég nem törik el.}{Játékos mezőre lép.}
\teszteset{StepInWaterWithScubaGear}{A játékos víz mezőre lép. A játékoson van búvárruha. A játékos kibírja a vízbe esést.}{Játékos mezőre lép.}
\teszteset{StepInWaterNaked}{A játékos víz mezőre lép. A játékoson nincs búvárruha.A játékos meg fog fagyni, ha nem mentik ki később.}{Játékos mezőre lép.}
\teszteset{RopeRescue}{A játékosnak van kötele. Egy játékos kiment egy másik játékost a vízből és a saját táblájára húzza.}{Játékos cselekszik}
\teszteset{EatFood}{A játékosnak van élelme. A játékos elfogyaszt 1 élelmet a zsebéből és nő a testhője.}{Játékos cselekszik.}
\teszteset{AssembleFlare}{A játékosnál van az összes rakéta darab. A játékos összeszereli azt.}{Játékos cselekszik.}
\teszteset{AssembleFlareFail}{A játékosnál nincs meg az összes rakéta darab és megpróbálja összeszerelni a pisztolyt.}{Játékos cselekszik.}
\teszteset{BuildIgloo}{Az eszkimó a táblára iglut épít.}{Tábla esemény teszt.}
\teszteset{BuildTent}{A játékos a táblára sátrat épít.}{Tábla esemény teszt.}
\teszteset{ExamineTile}{A sarkkutató a táblát vizsgálja.}{Tábla esemény teszt.}
\teszteset{TurnOnStableIce}{Egy játék kör eltelik úgy, hogy a játékos a stabil jégen áll.}{Játék esemény teszt.}
\teszteset{TurnInWaterNaked}{Egy játék kör eltelik úgy, hogy a játékos vízben áll. A játékoson van búvárruha.}{Játék esemény teszt.}
\teszteset{TurnInWaterWithScubaGear}{Egy játék kör eltelik úgy, hogy a játékos vízben áll. A játékoson nincs búvárruha.}{Játék esemény teszt.}
\teszteset{ChillStormIgloo}{Egy játék kör eltelik, úgy, hogy a játékost hóvihar éri. A játékos igluban van.}{Játék esemény teszt.}
\teszteset{ChillStormTent}{Egy játék kör eltelik, úgy, hogy a játékost hóvihar éri. A játékos sátorban van.}{Játék esemény teszt.}
\teszteset{ChillStormBareIce}{Egy játék kör eltelik, úgy, hogy a játékost hóvihar éri. A játékos szabad ég alatt van.}{Játék esemény teszt.}
\teszteset{TentBreaking}{A táblán a sátor eltörik a kör után.}{Játék esemény teszt.}
\teszteset{PolarBearMoving}{A medve véletlen irányba lép. Üres mezőre lép.}{Játék esemény teszt.}
\teszteset{PolarBearAttack}{A medve véletlen irányba lép. A mezőn van valaki.}{Játék esemény teszt.}
\teszteset{PolarBearAttackTent}{A medve véletlen irányba lép. A mezőn sátorban van valaki.}{Játék esemény teszt.}
\teszteset{PolarBearAttackIgloo}{A medve véletlen irányba lép. A mezőn igluban van valaki.}{Játék esemény teszt.}

\section{Tesztelést támogató segéd- és fordítóprogramok specifikálása}
%\comment{Specifikálni kell a tesztelést támogató segédprogramokat.}

