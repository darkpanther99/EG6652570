% Szglab4
% ===========================================================================
%
\chapter{Prototípus koncepciója}

\thispagestyle{fancy}

\section{Prototípus interface-definíciója}
\comment{Definiálni kell a teszteket leíró nyelvet. Külön figyelmet kell fordítani arra, hogy ha a rendszer véletlen elemeket is tartalmaz, akkor a véletlenszerűség ki-bekapcsolható legyen, és a program determinisztikusan is tesztelhető legyen.}

\subsection{Az interfész általános leírása}
\comment{A protó (karakteres) input és output felületeit úgy kell kialakítani, hogy az input fájlból is vehető legyen illetőleg az output fájlba menthető legyen, vagyis kommunikációra csak a szabványos be- és kimenet használható.}

\subsection{Bemeneti nyelv}
\comment{Definiálni kell a teszteket leíró nyelvet. Külön figyelmet kell fordítani arra, hogy ha a rendszer véletlen elemeket is tartalmaz, akkor a véletlenszerűség ki-bekapcsolható legyen, és a program determinisztikusan is futtatható legyen. A szálkezelést is tesztelhető, irányítható módon kell megoldani.}

\begin{itemize}
\item Parancs1
	\begin{itemize}
	\item Leírás:
	\item Opciók:
	\end{itemize}
\item Parancs2
	\begin{itemize}
	\item Leírás:
	\item Opciók:
	\end{itemize}

\end{itemize}

\comment{Ha szükséges, meg kell adni a konfigurációs (pl. pályaképet megadó) fájlok nyelvtanát is.}

\subsection{Kimeneti nyelv}
\comment{Egyértelműen definiálni kell, hogy az egyes bemeneti parancsok végrehajtása után előálló állapot milyen formában jelenik meg a szabványos kimeneten.}

\section{Összes részletes use-case}
\comment{A use-case-eknek a részletezettsége feleljen meg a kezelői felületnek, azaz a felület elemeire kell hivatkozniuk.
Alábbi táblázat minden use-case-hez külön-külön.}

\begin{figure}[h]
\begin{center}
%\includegraphics[width=17cm]{chapters/chapter07/example.pdf}
\caption{x}
\label{fig:ProtoUseCase}
\end{center}
\end{figure}

\usecase{...}{...}{...}{...}

\section{Tesztelési terv}
\comment{A tesztelési tervben definiálni kell, hogy a be- és kimeneti fájlok egybevetésével miként végezhető el a program tesztelése. Meg kell adni teszt forgatókönyveket. Az egyes teszteket elég informálisan, szabad szövegként leírni. Teszt-esetenként egy-öt mondatban. Minden teszthez meg kell adni, hogy mi a célja, a proto mely funkcionalitását, osztályait stb. teszteli. Az alábbi táblázat minden teszt-esethez külön-külön elkészítendő.}

\teszteset{PickUpFood}{A játékos az adott mezőre lép. A mezőn élelem található. A játékos felveszi és a saját zsebébe rakja az ételt.}{Item felvétel teszt.}
\teszteset{PickUpPart}{A játékos adott mezőre lép. A mezőn rakéta pisztoly darab található. A játékos felveszi és a megfelelő zsebébe rakja.}{Item felvétel teszt.}
\teszteset{PickUpShovel}{A játékos adott mezőre lép. A mezőn ásó található. A játékos felvesz és a zsebébe rakja. A játékos ez utána már tud ásni később.}{Item felvétel teszt.}
\teszteset{PickUpBreakableShovel}{A játékos adott mezőre lép. A mezőn törékeny ásó található. A játékos felvesz és a zsebébe rakja. A játékos ez utána már tud ásni később.}{Item felvétel teszt.}
\teszteset{PickUpRope}{A játékos adott mezőre lép. A mezőn kötél található. A játékos felvesz és a zsebébe rakja. A játékos ez utána már ki tudja menteni a csapattársait.}{Item felvétel teszt.}
\teszteset{PickUpScubaGear}{A játékos adott mezőre lép. A mezőn búvárruha található. A játékos felvesz és a zsebébe rakja. A játékos ez utána már képes túlélni a vízben.}{Item felvétel teszt.}
\teszteset{PickUpTent}{A játékos adott mezőre lép. A mezőn sátor található. A játékos felvesz és a zsebébe rakja. A játékos ez utána már képes sátrat építeni egy mezőre.}{Item felvétel teszt.}
\teszteset{BareHandsDig}{A játékos azon a mezőn ahol éppen áll havat lapátol. A hómennyiség az adott mezőn csökken}{Hó lapátolás.}
\teszteset{ShovelDig}{A játékos azon a mezőn ahol éppen áll havat lapátol, van nála lapát. A hómennyiség az adott mezőn csökken}{Hó lapátolás.}
\teszteset{BreakingShovelDig}{A játékos azon a mezőn ahol éppen áll havat lapátol, van nála törékeny lapát. A hómennyiség az adott mezőn csökken. A lapát még nem törik el.}{Hó lapátolás.}
\teszteset{BreakingShovelDig}{A játékos azon a mezőn ahol éppen áll havat lapátol, van nála törékeny lapát. A hómennyiség az adott mezőn csökken. A lapát eltörik}{Hó lapátolás.}
\teszteset{StepOnStableIce}{A játékos stabil jégre lép.}{Játékos mezőre lép.}
\teszteset{StepOnUnstableIceWithScubaGearBreaking}{A játékos instabil jégre lép. A játékoson van búvárruha. A jég eltörik, mert nem bírja el. Minden játékos vízbe esik aki ott volt.}{Játékos mezőre lép.}
\teszteset{StepOnUnstableIceWithScubaGearCanHold}{A játékos instabil jégre lép. A játékoson van búvárruha. A jég nem törik el.}{Játékos mezőre lép.}
\teszteset{StepOnUnstableIceNakedBreaking}{A játékos instabil jégre lép. A játékoson nincs búvárruha. A jég eltörik, mert nem bírja el. Minden játékos vízbe esik aki ott volt.}{Játékos mezőre lép.}
\teszteset{StepOnUnstableIceNakedCanHold}{A játékos instabil jégre lép. A játékoson nincs búvárruha. A jég nem törik el.}{Játékos mezőre lép.}
\teszteset{StepInWaterWithScubaGear}{A játékos víz mezőre lép. A játékoson van búvárruha. A játékos kibírja a vízbe esést.}{Játékos mezőre lép.}
\teszteset{StepInWaterNaked}{A játékos víz mezőre lép. A játékoson nincs búvárruha.A játékos meg fog fagyni, ha nem mentik ki később.}{Játékos mezőre lép.}
\teszteset{RopeRescue}{A játékosnak van kötele. Egy játékos kiment egy másik játékost a vízből és a saját táblájára húzza.}{Játékos cselekszik}
\teszteset{EatFood}{A játékosnak van élelme. A játékos elfogyaszt 1 élelmet a zsebéből és negnő a testhője.}{Játékos cselekszik.}
\teszteset{AssembleFlare}{A játékosnál van az összes rakéta darab. A játékos összeszereli azt.}{Játékos cselekszik.}
\teszteset{AssembleFlareFail}{A játékosnál nincs meg az összes rakéta darab és megbpróbálja összeszerelni a pisztolyt.}{Játékos cselekszik.}
\teszteset{BuildIgloo}{Az eszkimó a táblára iglut épít.}{Tábla esemény teszt.}
\teszteset{BuildTent}{A játékos a táblára sátrat épít.}{Tábla esemény teszt.}
\teszteset{ExamineTile}{A sarkkutató a táblát vizsgálja.}{Tábla esemény teszt.}
\teszteset{TurnOnStableIce}{Egy játék kör eltelik úgy, hogy a játékos a stabil jégen áll.}{Játék esemény teszt.}
\teszteset{TurnInWaterNaked}{Egy játék kör eltelik úgy, hogy a játékos vízben áll. A játékoson van búvárruha.}{Játék esemény teszt.}
\teszteset{TurnInWaterWithScubaGear}{Egy játék kör eltelik úgy, hogy a játékos vízben áll. A játékoson nincs búvárruha.}{Játék esemény teszt.}
\teszteset{ChillStormIgloo}{Egy játék kör eltelik, úgy, hogy a játékost hóvihar éri. A játékos igluban van.}{Játék esemény teszt.}
\teszteset{ChillStormTent}{Egy játék kör eltelik, úgy, hogy a játékost hóvihar éri. A játékos sátorban van.}{Játék esemény teszt.}
\teszteset{ChillStormBareIce}{Egy játék kör eltelik, úgy, hogy a játékost hóvihar éri. A játékos szabad ég alatt van.}{Játék esemény teszt.}
\teszteset{TentBreaking}{A táblán a sátor eltörik a kör után.}{Játék esemény teszt.}
\teszteset{PolarBearMoving}{A medve véletlen irányba lép. Üres mezőre lép.}{Játék esemény teszt.}
\teszteset{PolarBearAttack}{A medve véletlen irányba lép. A mezőn van valaki.}{Játék esemény teszt.}
\teszteset{PolarBearAttackTent}{A medve véletlen irányba lép. A mezőn sátorban van valaki.}{Játék esemény teszt.}
\teszteset{PolarBearAttackIgloo}{A medve véletlen irányba lép. A mezőn igluban van valaki.}{Játék esemény teszt.}

\section{Tesztelést támogató segéd- és fordítóprogramok specifikálása}
\comment{Specifikálni kell a tesztelést támogató segédprogramokat.}

