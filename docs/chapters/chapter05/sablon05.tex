% Szglab4
% ===========================================================================
%
\chapter{Szkeleton tervezése}

\thispagestyle{fancy}

\section{A szkeleton modell valóságos use-case-ei}
\comment{A szkeletonnak, mint önálló programnak a működésével kapcsolatos use-case-ek.}

\subsection{Use-case diagram}

\begin{figure}[h]
\begin{center}
%\includegraphics[width=17cm]{chapters/chapter05/example.pdf}
\caption{x}
\label{fig:SzkeletonUseCase}
\end{center}
\end{figure}

\subsection{Use-case leírások}
% TODO: Valaki latexhez értő írja ezeket át itemize-ra!
\usecase{Test PickUp Shovel}{Játékos lapátot vesz fel.}{Skeleton}{
	1. Eszkimó hóval nem rendelkező jégtáblán áll, amin egy lapát található.\newline
	2. Az eszkimó energiája csökken.\newline
	2.A Az eszkimó fáradt és nem tud tárgyat felvenni.\newline
	3. Az eszkimó felveszi a lapátot.\newline
	4. A lapát bekerül az eszkimó tárgyai közé és a megfelelő stratégiája helyére is.
}
\usecase{Test PickUp Food}{Játékos ételt vesz fel.}{Skeleton}{
	1. Eszkimó hóval nem rendelkező jégtáblán áll, amin egy élelem található.\newline
	2. Az eszkimó energiája csökken.\newline
	2.A Az eszkimó fáradt és nem tud tárgyat felvenni.\newline
	3. Az eszkimó felveszi az élelmet
	4. Az élelem bekerül az eszkimó tárgyai közé és a kajatárolójába is.
}
\usecase{Test PickUp Part}{Játékos alkatrészt vesz fel.}{Skeleton}{
	1. Eszkimó hóval nem rendelkező jégtáblán áll, amin egy rakéta alkatrész található.\newline
	2. Az eszkimó energiája csökken.\newline
	2.A Az eszkimó fáradt és nem tud tárgyat felvenni.\newline
	3. Az eszkimó felveszi az alkatrészt.\newline
	4. Az alkatrész bekerül az eszkimó tárgyai közé és a rakétadarabtárolójába is.
}
\usecase{Test PickUp Rope}{Játékos kötelet vesz fel.}{Skeleton}{
	1. Eszkimó hóval nem rendelkező jégtáblán áll, amin egy kötél található.\newline
	2. Az eszkimó energiája csökken.\newline
	2.A Az eszkimó fáradt és nem tud tárgyat felvenni.\newline
	3. Az eszkimó felveszi a kötelet.\newline
	4. A kötél bekerül az eszkimó tárgyai közé és a megfelelő stratégiája helyére is.
}
\usecase{Test PickUp ScubaGear}{Játékos búváruhát vesz fel.}{Skeleton}{
	1. Eszkimó hóval nem rendelkező jégtáblán áll, amin egy búvárruha található.\newline
	2. Az eszkimó energiája csökken.\newline
	2.A Az eszkimó fáradt és nem tud tárgyat felvenni.\newline
	3. Az eszkimó felveszi a búvárruhát.\newline
	4. A búvárruha bekerül az eszkimó tárgyai közé és a megfelelő stratégiája helyére is.
}
\usecase{Test BareHandsDig}{Játékos üres kézzel havat lapátol.}{Skeleton}{
	1. Eszkimó hóval rendelkező jégtáblán áll.\newline
	2. Az eszkimó energiája csökken.\newline
	2.A Az eszkimó fáradt és nem tud tárgyat felvenni.\newline
	3. Az eszkimó a lapátja segítségével 2 havat ellapátol a jégtábláról.
} 
\usecase{Test ShovelDig}{Játékos lapáttal havat lapátol.}{Skeleton}{
	1. Eszkimó hóval rendelkező jégtáblán áll.\newline
	2. Az eszkimó energiája csökken.\newline
	2.A Az eszkimó fáradt és nem tud tárgyat felvenni.\newline
	3. Az eszkimó a keze segítségével 1 havat ellapátol a jégtábláról.
} 
\usecase{Test StepOnIce}{Játékos jégre lép.}{Skeleton}{
	1. Eszkimó jégtáblán áll és van előtte egy másik jégtábla.\newline
	2. Az eszkimó energiája csökken.\newline
	2.A Az eszkimó fáradt és nem tud előrelépni.\newline
	3. Az eszkimó előrelép.
}
\usecase{Test StepOnUnstableIce WithScubaGear}{Búvárruhás játékos instabil jégre lép.}{Skeleton}{
	1. Búvárruhás eszkimó jégtáblán áll és van előtte egy másik jégtábla, ami csak egy főt bír el, és áll rajta egy másik eszkimó.\newline
	2. Az eszkimó energiája csökken.\newline
	2.A. Alter: Az eszkimó fáradt és nem tud előrelépni.\newline
	3. Az eszkimó előrelép.\newline
	4. A jégtábla beszakad.\newline
	5. A búvárruha megvédi az eszkimót a hideg víztől.
}
\usecase{Test StepOnUnstableIce Naked}{Játékos instabil jégre lép.}{Skeleton}{
	1. Eszkimó jégtáblán áll és van előtte egy másik jégtábla, ami csak egy főt bír el, és áll rajta egy másik eszkimó.\newline
	2. Az eszkimó energiája csökken.\newline
	2.A Az eszkimó fáradt és nem tud előrelépni.\newline
	3. Az eszkimó előrelép.\newline
	4. A jégtábla beszakad.\newline
	5. Az eszkimó elkezd fuldokolni a hideg vízben.
}
\usecase{Test StepInHole WithScubaGear}{Búvárruhás játékos lyukba esik.}{Skeleton}{
	1. Búvárruhás eszkimó jégtáblán áll és van előtte egy hóval fedett lyuk.\newline
	2. Az eszkimó energiája csökken.\newline
	2.A Az eszkimó fáradt és nem tud előrelépni.\newline
	3. Az eszkimó előrelép.\newline
	4. A hó beszakad.\newline
	5. A búvárruha megvédi az eszkimót a hideg víztől.
}
\usecase{Test StepInHole Naked}{Játékos lyukba esik.}{Skeleton}{
	1. Búvárruhás eszkimó jégtáblán áll és van előtte egy hóval fedett lyuk.\newline
	2. Az eszkimó energiája csökken.\newline
	2.A Az eszkimó fáradt és nem tud előrelépni.\newline
	3. Az eszkimó előrelép.\newline
	4. A hó beszakad.\newline
	5. Az eszkimó elkezd fuldokolni a hideg vízben.
}
\usecase{Test RopeRescue}{...}{Skeleton}{...}
\usecase{Test EatFood}{...}{Skeleton}{...}
\usecase{Test AssebleFlare}{...}{Skeleton}{...}
\usecase{Test BuildIgloo}{...}{Skeleton}{...}
\usecase{Test ExamineTile}{...}{Skeleton}{...}
\usecase{Test Turn}{...}{Skeleton}{...}
\usecase{Test Turn WithScubaGear}{...}{Skeleton}{...}
\usecase{Test Turn Naked}{...}{Skeleton}{...}
\usecase{Test ChillStorm Igloo }{...}{Skeleton}{...}
\usecase{Test ChillStorm BareIce}{...}{Skeleton}{...}

\section{A szkeleton kezelői felületének terve, dialógusok}
\comment{A szkeleton által elfogadott bemenetek , valamint a szöveges konzolon megjelenő kimenetek. A kiemenet formátuma olyan kell legyen, ami alapján a működés összevethető a korábbi szekvencia-diagramokkal.}

\section{Szekvencia diagramok a belső működésre}
\comment{A szkeletonban implementált szekvenciadiagramok. Tipikusan egy use-case egy diagram. Ezek megegyezhetnek a korábban specifikált diagramokkal, de az egyes életvonalakat (lifeline) egyértelműen a szkeletonban példányosított objektumokhoz kell tudni kötni. Azt kell megjeleníteni, hogy a szkeletonban létrehozott objektumok egymással hogyan fognak kommunikálni.}

\section{Kommunikációs diagramok}
\comment{A szkeletonban, az egyes szkeleton-use-case-ek futása során létrehozott objektumok és kapcsolataik bemutatására szolgáló diagramok. Ezek alapján valósítják meg a szkeleton fejlesztői az inicializáló kódrészleteket.}
