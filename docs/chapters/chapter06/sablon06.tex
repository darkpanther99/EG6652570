\chapter{Szkeleton beadás}

\thispagestyle{fancy}

\section{Fordítási és futtatási útmutató}
%\comment{A feltöltött program fordításával és futtatásával kapcsolatos útmutatás. Ennek tartalmaznia kell leltárszerűen az egyes fájlok pontos nevét, méretét byte-ban, keletkezési idejét, valamint azt, hogy a fájlban mi került megvalósításra.}

\subsection{Fájllista}

\begin{fajllista}

%\fajl{név}{méret}{keletkezés ideje}{leírás}
\fajl{build.bat}{0,53KB}{2020.03.28.~14:17~}{Fordító script}
\fajl{Logger.java}{4,44KB}{2020.03.20.~23:54~}{Naplózó osztály forráskódja}
\fajl{Main.java}{13,4KB}{2020.03.15.~18:33~}{A konzolalkalmazás belépési pontja}

\fajl{BareHands.java}{0,77KB}{2020.03.15.~18:33~}{BareHands osztály forráskódja}
\fajl{BareIce.java}{0,67KB}{2020.03.15.~18:33~}{BareIce osztály forráskódja}
\fajl{CantRescue.java}{0,73KB}{2020.03.15.~18:33~}{CantRescue osztály forráskódja}
\fajl{ChillStormStrategy.java}{0,28KB}{2020.03.15.~18:33~}{ChillStormStrategy osztály forráskódja}
\fajl{ChillWaterStrategy.java}{0,3KB}{2020.03.15.~18:33~}{ChillWaterStrategy osztály forráskódja}
\fajl{DigStrategy.java}{0,27KB}{2020.03.15.~18:33~}{DigStrategy osztály forráskódja}
\fajl{Direction.java}{0,53KB}{2020.03.28.~20:51~}{Direction osztály forráskódja}
\fajl{DryLand.java}{0,56KB}{2020.03.15.~18:33~}{DryLand osztály forráskódja}
\fajl{Empty.java}{0,37KB}{2020.03.15.~18:33~}{Empty osztály forráskódja}
\fajl{Eskimo.java}{0,51KB}{2020.03.15.~18:33~}{Eskimo osztály forráskódja}
\fajl{Food.java}{0,43KB}{2020.03.15.~18:33~}{Food osztály forráskódja}
\fajl{FoodStore.java}{0,92KB}{2020.03.15.~18:33~}{FoodStore osztály forráskódja}
\fajl{Game.java}{7,51KB}{2020.03.15.~18:33~}{Game osztály forráskódja}
\fajl{Igloo.java}{0,44KB}{2020.03.15.~18:33~}{Igloo osztály forráskódja}
\fajl{Item.java}{0,24KB}{2020.03.15.~18:33~}{Item osztály forráskódja}
\fajl{Naked.java}{0,47KB}{2020.03.15.~18:33~}{Naked osztály forráskódja}
\fajl{Part.java}{0,46KB}{2020.03.15.~18:33~}{Part osztály forráskódja}
\fajl{PartStore.java}{1,44KB}{2020.03.15.~18:33~}{PartStore osztály forráskódja}
\fajl{Player.java}{11,18KB}{2020.03.15.~18:33~}{Player osztály forráskódja}
\fajl{PolarExplorer.java}{0,68KB}{2020.03.15.~18:33~}{PolarExplorer osztály forráskódja}
\fajl{RescueStrategy.java}{0,42KB}{2020.03.15.~18:33~}{RescueStrategy osztály forráskódja}
\fajl{Rope.java}{0,51KB}{2020.03.15.~18:33~}{Rope osztály forráskódja}
\fajl{RopeRescue.java}{1,12KB}{2020.03.15.~18:33~}{RopeRescue osztály forráskódja}
\fajl{ScubaGear.java}{0,56KB}{2020.03.15.~18:33~}{ScubaGear osztály forráskódja}
\fajl{ScubaWearing.java}{0,53KB}{2020.03.15.~18:33~}{ScubaWearing. osztály forráskódja}
\fajl{Sea.java}{0,59KB}{2020.03.15.~18:33~}{Sea osztály forráskódja}
\fajl{Shovel.java}{0,51KB}{2020.03.15.~18:33~}{Shovel osztály forráskódja}
\fajl{ShovelDig.java}{1,13KB}{2020.03.15.~18:33~}{ShovelDig osztály forráskódja}
\fajl{Tile.java}{4,81KB}{2020.03.15.~18:33~}{Tile osztály forráskódja}
\fajl{WaterResistanceStrategy.java}{0,38KB}{2020.03.15.~18:33~}{WaterResistanceStrategy osztály forráskódja}

\end{fajllista}

\subsection{Fordítás}
%\comment{A fenti listában szereplő forrásfájlokból milyen műveletekkel lehet a bináris, futtatható kódot előállítani. Az előállításhoz csak a 2. Követelmények c. dokumentumban leírt környezetet szabad előírni.}
A fordítást a skeleton\textbackslash{}build.bat script végzi. Létrehozza a skeleton\textbackslash{}out\textbackslash{}skeleton.jar fájlt, és el is indítja.

\lstset{escapeinside=`', xleftmargin=10pt, frame=single, basicstyle=\ttfamily\footnotesize, language=sh}
\begin{lstlisting}
skeleton\build
\end{lstlisting}

\subsection{Futtatás}
%\comment{A futtatható kód elindításával kapcsolatos teendők leírása. Az indításhoz csak a 2. Követelmények c. dokumentumban leírt környezetet szabad előírni.}

\lstset{escapeinside=`', xleftmargin=10pt, frame=single, basicstyle=\ttfamily\footnotesize, language=sh}
\begin{lstlisting}
java -jar skeleton\out\skeleton.jar
\end{lstlisting}

\section{Értékelés}
%\comment{A projekt kezdete óta az értékelésig eltelt időben tagokra bontva, százalékban.}

Az értékelést a tagok elolvasták és elfogadták.

\begin{ertekeles}
\tag{Glávits}{24}{-}
\tag{Kiss}{22}{-}
\tag{Konrád}{18}{-}
\tag{Lant}{18}{-}
\tag{Máté}{18}{-}
\end{ertekeles}

