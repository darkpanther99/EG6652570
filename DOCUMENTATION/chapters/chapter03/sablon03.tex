% Szoftprojlab
% ===========================================================================
%
\chapter{Analízis modell kidolgozása 1}

\thispagestyle{fancy}

\section{Objektum katalógus}

\subsection{Játékos}
Három vagy több van belőle. Körökre bontva teszik a dolgukat. Saját körükben tudnak mozogni, különböző tárgyakat használni vagy a speciális képességüket használni. A játék megnyeréséhez szükséges rakétapisztoly alkatrészek összegyűjétse a feladatuk. Ha vízbe esnek, vagy kihűlnek akkor a játéknak vége.

\subsection{Jégtábla}
Ilyenek alkotják a játékos számára a játékteret, ezeken lehet mozogni. Jégtáblák tartalmazhatnak tárgyakat amelyeket ki lehet ásni. Az instabil jégtábla képes vízbe ejteni a rajta állókat, ha túl sokan vannak. A jégtáblán lehet hó. Néha lehet rajta hóvihar, mely csökkenti a rajta állók testhőjét

\subsection{Kötél}
Ennek segítésével ki lehet húzni egy vízbe esett játékost.

\subsection{Búvárruha}
A játékos képes a vízben is mozogni vele, illetve nem veszít testhőt ha vízben tartózkodik.

\subsection{Lapát}
Segítségével 2 egységnyi hó takarítható el, egy egység munkával.

\subsection{Élelem}
Ha a játékos elfogyasztja, a testhője 1-el megnő.

\subsection{Rakétapisztoly Alkatrész}
A játékban 3 darab ilyen megtalálása vezet a játék sikeres befejezéséhez. Az összeszereléshez mindháromnak egy helyen kell lennie.

\subsection{Iglu}
Eszkimó (Játékos) képes építeni, itt átvészelhetőek a hóviharok.

\newpage
\section{Statikus struktúra diagramok}
%\comment{Az előző alfejezet osztályainak kapcsolatait és publikus metódusait bemutató osztálydiagram(ok). Tipikus hibalehetőségek: csillag-topológia, szigetek.}

%\begin{figure}[h]
%\begin{center}
%\includegraphics[width=17cm]{chapters/chapter03/example.pdf}
%\caption{x}
%\label{fig:example1}
%\end{center}
%\end{figure}


\section{Osztályok leírása}
%\comment{Az előző alfejezetben tárgyalt objektumok felelősségének formalizálása attribútumokká, metódusokká. Csak publikus metódusok szerepelhetnek. Ebben az alfejezetben megjelennek az interfészek, az öröklés, az absztrakt osztályok. Segédosztályokra még mindig nincs szükség. Az osztályok ABC sorrendben kövessék egymást. Interfészek esetén az Interfészek, Attribútumok pontok kimaradnak.}

\subsection{BareHands}
\begin{itemize}
\item A játékos így ás, ha nincs ásója.\\

\item Interfészek:
	\begin{itemize}
		\item DigStrategy
	\end{itemize}

\item Metódusok:
	\begin{itemize}
		\item bool Dig(Tile t): Csökkenti a tile-on található hó mennyiségét (int)
	\end{itemize}
\end{itemize}

\subsection{BareIce}
\begin{itemize}
\item A jégtáblán nincs védelem a vihar elől.\\

\item Interfészek:

	\begin{itemize}
		\item ChillStormStrategy
	\end{itemize}

\item Metódusok:
	\begin{itemize}
		\item void Chill(Tile t): Táblán alló játékosok testhője csökken.
	\end{itemize}
\end{itemize}

\subsection{CantRescue}
\begin{itemize}
	\item A játékos nem tudja kihúzni a csapattársát.
	
\item Interfészek:
\begin{itemize}
	\item RescueStrategy
\end{itemize}

\item Metódusok:
\begin{itemize}
	\item void Rescue(Tile water, Tile land): üres
\end{itemize}
\end{itemize}

\subsection{ChillStormStrategy}
\begin{itemize}
	\item A jégtábla így hűti viharban a játékosokat.

	\item Metódusok:
	\begin{itemize}
		\item abstract void Chill(Tile t)
	\end{itemize}
\end{itemize}

\subsection{ChillWaterStrategy}
\begin{itemize}
	\item A jégtábla így hűti a vízbe esett játékosokat.\\
	
\item Metódusok:
\begin{itemize}
	\item abstract void Chill(Tile t)
\end{itemize}
\end{itemize}

\subsection{DigStrategy}
\begin{itemize}
	\item A játékos így ás.\\
	
\item Metódusok:
\begin{itemize}
	\item abstract bool Dig(Tile t)
\end{itemize}
\end{itemize}

\subsection{DryLand}
\begin{itemize}
	\item A szárazföld nem hűti a játékosokat.\\
	
\item Interfészek:
	\begin{itemize}
		\item ChillWaterStrategy
	\end{itemize}
\item Metódusok:
\begin{itemize}
	\item void Chill(Tile t): üres
\end{itemize}
\end{itemize}

\subsection{Empty}
\begin{itemize}
	\item Nincs jégbe fagyott tárgy.
	
\item Interfészek:
	\begin{itemize}
		\item GiveItemStrategy
	\end{itemize}
\item Metódusok
\begin{itemize}
	\item void GiveTo(Player p): üres
\end{itemize}
\end{itemize}

\subsection{Eskimo}
\begin{itemize}
	\item Játékos osztály.
	
	\item Ősosztályok:
	\begin{itemize}
		\item Player
	\end{itemize}
\item Metódusok:
\begin{itemize}
	\item void BuildIgloo(): Épít egy iglut a mezőre, amin áll.
\end{itemize}
\end{itemize}

\subsection{Food}
\begin{itemize}
	\item Élelem, amit a játékos meg tud enni, hogy növelje a testhőjét.
	
\item Interfészek:
\begin{itemize}
	\item GiveItemStrategy
\end{itemize}

\item Metódusok:
\begin{itemize}
	\item void GiveTo(Player p): A játékos kap egy élelmet.
\end{itemize}
\end{itemize}

\subsection{FoodStore}
\begin{itemize}
	\item A játékos ebben a zsebben tárolja az élelmet.

\item Attribútumok:

\begin{itemize}
	\item count: int: Hány élelem van a játékosnál

\end{itemize}
\item Metódusok:
\begin{itemize}
	\item void feed(Player p): Játékos testhője megnő.
\end{itemize}
\end{itemize}

\subsection{Game}
\begin{itemize}
	\item Interface a Model és a Controller között. A játékmesterhez tartozó működést valósítja meg.
	
\item Attribútumok:

\begin{itemize}
	\item players: Player[3..*]: Tárolja a játékosokat
	\item icefield: Tile[1..*]: Tárolja a pályát alkotó elemeket
\end{itemize}
\item Metódusok:
\begin{itemize}
	\item Tile CreateIce(): Létrehoz egy jégtáblát. Ez a metódus az init szekvencia része.
	\item Tile CreateUnstableIce(): Létrehoz egy instabil jégtáblát. Ez a metódus az init szekvencia része.
	\item Tile CreateSea(): Létrehoz egy vizet. Ez a metódus az init szekvencia része.
	\item Tile CreateHole(): Létrehoz egy lyukat: olyan vizet amit hó fed. Ez a metódus az init szekvencia része.
	\item Player CreateEskimo(): Létrehoz egy eszkimó játékost. Ez a metódus az init szekvencia része.
	\item Player CreatePolarExplorer(): Létrehoz egy sarkkutató játékost. Ez a metódus az init szekvencia része.
	\item void GameOver(): Ha vége a játéknak, szól a Controllernek, hogy vesztettünk. Külső metódus.
	\item void Turn(): Ezt a metódust a Controller hívja körönként. 
	\item void Victory(): Ha vége a játéknak, szól a Controllernek, hogy nyertünk. Külső metódus.
\end{itemize}
\end{itemize}

\subsection{Igloo}
\begin{itemize}
	\item Ezen a jégtáblán iglu áll, a játékosok védve vannak a vihartól.
	
\item Interfészek:
\begin{itemize} 
	\item ChillStromStrategy
\end{itemize}
\item Metódusok:
\begin{itemize}
	\item void Chill(Tile t): üres
\end{itemize}
\end{itemize}

\subsection{Naked}
\begin{itemize}
	\item A játékos védtelen a hideg vízzel szemben.
	
\item Interfészek:
\begin{itemize}
	\item WaterResistanceStrategy
\end{itemize}
\item Metódusok:
\begin{itemize}
	\item void Chill(Player p): Játékosnak nincsen ereje a vízben úszni búvárruha nélkül.
\end{itemize}
\end{itemize}

\subsection{Part}
\begin{itemize}
	\item Jégbefagyott alkatrész.

\item Interfészek:
\begin{itemize}
	\item GiveItemStrategy
\end{itemize}

\item Metódusok:
\begin{itemize}
	\item void GiveTo(Player p): A játékos tárolójába kerül egy darab a rakétapisztolyból.
\end{itemize}
\end{itemize}

\subsection{PartStore}
\begin{itemize}
	\item A játékos ebben a zsebben tárolja az alkatrészeket.
	
\item Attribútumok:
\begin{itemize}
	\item count: int: Hány darab alkatrész van belőle a játékosnál?
\end{itemize}
\item Metódusok:
\begin{itemize}
	\item void Gain(PartStore ps): Átveszi az alkatrészeket.
	\item void Gain(int n): Megnő az alkatrészek száma ami a játékosnál van.
\end{itemize}
\end{itemize}

\subsection{Player}
\begin{itemize}
	\item Játékos osztály, amit a felhasználó irányít a grafikus felületen keresztül.

\item Attribútumok:
\begin{itemize}
	\item bodyTemp: int: Jelzi a játékos jelenlegi hőmérsékletét, ha 0 akkor megfagy $\rightarrow$ játék vége.
	\item currentTile: Tile: A játékos ismeri a mezőt amin éppen áll.
	\item digStrategy: DigStrategy: Eldönti hogyan képes ásni a játékos.
	\item energy: int: Számlálja mennyit mozogott az adott körben a játékos.
	\item foodStore: FoodStore: Tárolja a játékos ételeit.
	\item partStore: PartStore: Tárolja a játékos rakéta alkatrészeit.
	\item rescueStrategy: RescueStrategy: Eldönti, hogy megmenthet egy játékos egy másikat a vízbeesés után.
	\item waterResistanceStrategy: WaterResistanceStrategy: Eldönti, hogy a játékos hogyan viselkedik vízbeesés esetén.
\end{itemize}
\item Metódusok:
\begin{itemize}
	\item void AssembleFlare(): Összerakja a játék végéhez szükséges rakéta pisztolyt. 1 munkaegység
	\item void Chill(): A testhő 1-el csökken, ha 0 alá megy GameOver.
	\item void DecrementEnergy(): Az energiát csökkentő helper metódus.
	\item void Dig(): Ezt a metódust a Controller hívja. A játékos havat ás. 1 munkaegység
	\item void EatFood(): Ezt a metódust a Controller hívja. A játékos eszik.
	\item void PickUp(): Ezt a metódust a Controller hívja. A játékos felvesz egy tárgyat. 1 munkaegység
	\item void PlaceOn(Tile t): Init szekvencia része. RopeRescue szekvencia része. Rárak egy játékost egy másik Tile-ra.
	\item void RescueTeammate(direction d): Ezt a metódust a Controller hívja. A játékos kiment egy másikat a vízből. 1 munkaegység
	\item void ResistWater(): A játékos testhője a WaterResistance szerint változik.
	\item void Step(): Ezt a metódust a Controller hívja. A játékos lép, ha van még hozzá elég energiája. 1 munkaegység
	\item void ToFoodStore(): Élelem megtalálásához helper metódus.
\end{itemize}
\end{itemize}


\subsection{PolarExplorer}
\begin{itemize}
	\item Játékos típus, akivel valaki játszhat
	
	\item Ősosztályok:
	\begin{itemize}
		\item Player
	\end{itemize}

\item Metódusok:
\begin{itemize}
	\item int Examine(direction d): A játékos megnézheti, hogy egy adott Tile-nak mennyi a teherbírása.
\end{itemize}
\end{itemize}

\subsection{RescueStrategy}
\begin{itemize}
	\item A játékos így húzza ki csapattársát a vízből.\

\item Metódusok:
\begin{itemize}
	\item abstract void Rescue(Tile water, Tile land): üres
\end{itemize}
\end{itemize}

\subsection{Rope}
\begin{itemize}
	\item Jégbe fagyott kötél.
\item Interfészek:
\begin{itemize}
	\item GiveItemStrategy
\end{itemize}
\item Metódusok
\begin{itemize}
	\item void GiveTo(Player p): Felrhuázza a játékost kötéllel.
\end{itemize}
\end{itemize}

\subsection{RopeRescue}
\begin{itemize}
	\item A játékos kihúzza csapattársát a vízből.
\item Interfészek:
\begin{itemize}
	\item RescueStrategy
\end{itemize}

\item Metódusok:
\begin{itemize}
	\item void Rescue(Tile water, Tile land): A vízben lévők közül egyvalaki rákerül a kihúzó játékos cellájára.
\end{itemize}
\end{itemize}

\subsection{ScubaGear}
\begin{itemize}
	\item Jégbe fagyott búvárruha.
\item Interfészek:
\begin{itemize}
	\item GiveItemStrategy
\end{itemize}
\item Metódusok:
\begin{itemize}
	\item void GiveTo(): Felruházza a játékost búvárruhával.
\end{itemize}
\end{itemize}

\subsection{Sea}
\begin{itemize}
	\item Ez a cella tenger, hűti a játékosokat.
\item Interfészek:
\begin{itemize}
	\item ChillWaterStrategy
\end{itemize}

\item Metódusok:
\begin{itemize}
	\item void Chill(Tile t): Minden rajta álló testhője csökken a WaterResistanceStrategy szerint.
\end{itemize}
\end{itemize}

\subsection{ShovelDig}
\begin{itemize}
	\item Egyszer lehet ásni vele fáradság nélkül is.
\item Interfészek:
\begin{itemize}
	\item DigStrategy
\end{itemize}

\item Attribútumok:
\begin{itemize}
	\item lastUsed: bool: Volt-e már használva a körben
\end{itemize}
\item Metódusok:
\begin{itemize}
	\item void Dig(Tile t): Csökkenti a tile-on található hó mennyiségét.
\end{itemize}
\end{itemize}

\subsection{Tile}
\begin{itemize}
	\item Cella, ilyenekből áll a jégmező ahol a játékosok játszanak.
\item Attribútumok:
\begin{itemize}
	\item chillStormStrategy: ChillStormStrategy: Eldönti, kinek változik a testhője vihar esetén.
	\item chillWaterStrategy: ChillWaterStrategy: Eldönti, kinek változik a testhője víz esetén.
	\item giveItemStrategy: GiveItemStrategy: Eldönti, milyen tárgyat vesz fel a találó.
	\item neighborTiles: Tile[*]: Szomszédos cellákat ismer.
	\item occupants: Player[*]: Rajta lévő játékosok.
	\item snow: int: Rajta lévő hómennyiség.
	\item weightLimit: int: Rajta lévő játékosok számának maximuma.
	
\end{itemize}
\item Metódusok:
\begin{itemize}
	\item void ChillStorm(): Ezt a metódust a Controller hívja viharban. Hűti a játékosokat, ha nincsenek igluban.
	\item void ChillWater(): Ezt a metódust a Controller hívja körönként. Hűti a játékosokat, ha ez a cella víz.
	\item void DecrementSnow(): A hómennyiséget csökkentő helper függvény.
	\item void GiveItem(Player): A játékos megkapja a tartalmazott tárgyat.
	\item Tile NeighborAt(direction): Visszaadja az adott irányban szomszédos cellát.
	\item StepOn(Player): Játékos rálép a cellára, ha többen vannak mint a korlát, a jégtábla átfordul.
	\item StepOff(Player): Járékos lelép a celláról.
\end{itemize}
\end{itemize}

\subsection{WaterResistanceStrategy}
\begin{itemize}
	\item Így reagál a játékos a hideg vízre.
\item Metódusok:
\begin{itemize}
	\item abstract void Chill(Player p): üres
\end{itemize}
\end{itemize}

\newpage
\section{Statikus struktúra diagramok}
%\comment{Az előző alfejezet osztályainak kapcsolatait és publikus metódusait bemutató osztálydiagram(ok). Tipikus hibalehetőségek: csillag-topológia, szigetek.}

%OSZTALY DIAGRAM IDE!!!!!

%\begin{figure}[h]
%\begin{center}
%\includegraphics[width=17cm]{chapters/chapter03/example.pdf}
%\caption{x}
%\label{fig:example1}
%\end{center}
%\end{figure}
\newpage
\section{Szekvencia diagramok}
%\comment{Inicializálásra, use-case-ekre, belső működésre. Konzisztens kell legyen az előző alfejezettel. Minden metódus, ami ott szerepel, fel kell tűnjön valamelyik szekvenciában. Minden metódusnak, ami szekvenciában szerepel, szereplnie kell a valamelyik osztálydiagramon.}
%find . -printf "\\\begin{figure}[H]\n\t\\\begin{center}\n\t\t\\\includegraphics[width=10cm]{chapters/chapter03/seqdiag/%f}\n\t\t\\\caption{aaa}\n\t\t\\\label{bbb}\n\t\\\end{center}\n\\\end{figure}\n"

\begin{figure}[H]
	\begin{center}
		\includegraphics[width=10cm]{chapters/chapter03/seqdiag/Game_init_player.png}
		\caption{Game.InitPlayer()}
		\label{fig:GameInitPlayer}
	\end{center}
\end{figure}
\begin{figure}[H]
	\begin{center}
		\includegraphics[width=10cm]{chapters/chapter03/seqdiag/Game_Turn.png}
		\caption{Game.Turn()}
		\label{fig:GameTurn}
	\end{center}
\end{figure}
\begin{figure}[H]
	\begin{center}
		\includegraphics[width=10cm]{chapters/chapter03/seqdiag/Game_CreateIce.png}
		\caption{Game.CreateIce()}
		\label{fig:GameCreateIce}
	\end{center}
\end{figure}
\begin{figure}[H]
	\begin{center}
		\includegraphics[width=10cm]{chapters/chapter03/seqdiag/Game_CreateUnstableIce.png}
		\caption{Game.CreateUnstableIce()}
		\label{fig:GameCreateUnstableIce}
	\end{center}
\end{figure}
\begin{figure}[H]
	\begin{center}
		\includegraphics[width=10cm]{chapters/chapter03/seqdiag/Game_CreateHole.png}
		\caption{Game.CreateHole()}
		\label{fig:GameCreateHole}
	\end{center}
\end{figure}
\begin{figure}[H]
	\begin{center}
		\includegraphics[width=10cm]{chapters/chapter03/seqdiag/Game_CreateSea.png}
		\caption{Game.CreateSea()}
		\label{fig:GameCreateSea}
	\end{center}
\end{figure}
\begin{figure}[H]
	\begin{center}
		\includegraphics[width=10cm]{chapters/chapter03/seqdiag/Game_CreatePolarExplorer.png}
		\caption{Game.CreatePolarExplorer()}
		\label{fig:GameCreatePolarExplorer}
	\end{center}
\end{figure}
\begin{figure}[H]
	\begin{center}
		\includegraphics[width=10cm]{chapters/chapter03/seqdiag/Game_CreateEskimo.png}
		\caption{Game.CreateEskimo()}
		\label{fig:GameCreateEskimo}
	\end{center}
\end{figure}
\begin{figure}[H]
	\begin{center}
		\includegraphics[width=10cm]{chapters/chapter03/seqdiag/Game_generate_item.png}
		\caption{Game.GenerateItem()}
		\label{fig:GameGenerateItem}
	\end{center}
\end{figure}
\begin{figure}[H]
	\begin{center}
		\includegraphics[width=10cm]{chapters/chapter03/seqdiag/Player_Step.png}
		\caption{Player.Step(direction)}
		\label{fig:PlayerStep}
	\end{center}
\end{figure}
\begin{figure}[H]
	\begin{center}
		\includegraphics[width=10cm]{chapters/chapter03/seqdiag/Player_Dig.png}
		\caption{Player.Dig()}
		\label{fig:PlayerDig}
	\end{center}
\end{figure}
\begin{figure}[H]
	\begin{center}
		\includegraphics[width=10cm]{chapters/chapter03/seqdiag/Player_PickUp.png}
		\caption{Player.PickUp()}
		\label{fig:PlayerPickUp}
	\end{center}
\end{figure}
\begin{figure}[H]
	\begin{center}
		\includegraphics[width=10cm]{chapters/chapter03/seqdiag/Player_PlaceOn.png}
		\caption{Player.PlaceOn(Tile)}
		\label{fig:PlayerPlaceOn}
	\end{center}
\end{figure}
\begin{figure}[H]
	\begin{center}
		\includegraphics[width=10cm]{chapters/chapter03/seqdiag/Player_EatFood.png}
		\caption{Player.EatFood()}
		\label{fig:PlayerEatFood}
	\end{center}
\end{figure}
\begin{figure}[H]
	\begin{center}
		\includegraphics[width=10cm]{chapters/chapter03/seqdiag/Player_RescueTeammate.png}
		\caption{Player.RescueTeammate(direction)}
		\label{fig:Player.RescueTeammate}
	\end{center}
\end{figure}
\begin{figure}[H]
	\begin{center}
		\includegraphics[width=10cm]{chapters/chapter03/seqdiag/Player_Chill.png}
		\caption{Player.Chill()}
		\label{fig:PlayerChill}
	\end{center}
\end{figure}
\begin{figure}[H]
	\begin{center}
		\includegraphics[width=10cm]{chapters/chapter03/seqdiag/Player_ResistWater.png}
		\caption{Player.ResistWater()}
		\label{fig:PlayerResistWater}
	\end{center}
\end{figure}
\begin{figure}[H]
	\begin{center}
		\includegraphics[width=10cm]{chapters/chapter03/seqdiag/Player_AssembleFlare.png}
		\caption{Player.AssembleFlare()}
		\label{fig:PlayerAssembleFlare}
	\end{center}
\end{figure}
\begin{figure}[H]
	\begin{center}
		\includegraphics[width=10cm]{chapters/chapter03/seqdiag/Player_assemble_flare.png}
		\caption{Player.AssembleFlare()}
		\label{fig:PlayerAssembleFlare2}
	\end{center}
\end{figure}
\begin{figure}[H]
	\begin{center}
		\includegraphics[width=10cm]{chapters/chapter03/seqdiag/Eskimo_BuildIgloo.png}
		\caption{Eskimo.BuildIgloo()}
		\label{fig:EskimoBuildIgloo}
	\end{center}
\end{figure}
\begin{figure}[H]
	\begin{center}
		\includegraphics[width=10cm]{chapters/chapter03/seqdiag/PolarExplorer_Examine.png}
		\caption{PolarExplorer.Examine(direction)}
		\label{fig:PolarExplorerExamine}
	\end{center}
\end{figure}
\begin{figure}[H]
	\begin{center}
		\includegraphics[width=10cm]{chapters/chapter03/seqdiag/Tile_StepOn.png}
		\caption{Tile.StepOn(Player)}
		\label{fig:TileStepOn}
	\end{center}
\end{figure}
\begin{figure}[H]
	\begin{center}
		\includegraphics[width=10cm]{chapters/chapter03/seqdiag/Tile_StepOff.png}
		\caption{Tile.StepOff(Player)}
		\label{fig:TileStepOff}
	\end{center}
\end{figure}
\begin{figure}[H]
	\begin{center}
		\includegraphics[width=10cm]{chapters/chapter03/seqdiag/Tile_GiveItem.png}
		\caption{Tile.GiveItem(Player)}
		\label{fig:TileGiveItem}
	\end{center}
\end{figure}
\begin{figure}[H]
	\begin{center}
		\includegraphics[width=10cm]{chapters/chapter03/seqdiag/Tile_ChillWater.png}
		\caption{Tile.ChillWater()}
		\label{fig:TileChillWater}
	\end{center}
\end{figure}
\begin{figure}[H]
	\begin{center}
		\includegraphics[width=10cm]{chapters/chapter03/seqdiag/Tile_ChillStorm.png}
		\caption{Tile.ChillStorm()}
		\label{fig:TileChillStorm}
	\end{center}
\end{figure}
\begin{figure}[H]
	\begin{center}
		\includegraphics[width=10cm]{chapters/chapter03/seqdiag/Naked_Chill.png}
		\caption{Naked.Chill(Player)}
		\label{fig:NakedChill}
	\end{center}
\end{figure}
\begin{figure}[H]
	\begin{center}
		\includegraphics[width=10cm]{chapters/chapter03/seqdiag/ScubaWearing_Chill.png}
		\caption{ScubaWearing.Chill(Player)}
		\label{fig:ScubaWearingChill}
	\end{center}
\end{figure}
\begin{figure}[H]
	\begin{center}
		\includegraphics[width=10cm]{chapters/chapter03/seqdiag/BareIce_Chill.png}
		\caption{BareIce.Chill()}
		\label{fig:BareIceChill}
	\end{center}
\end{figure}
\begin{figure}[H]
	\begin{center}
		\includegraphics[width=10cm]{chapters/chapter03/seqdiag/DryLand_Chill.png}
		\caption{DryLand.Chill(Tile)}
		\label{fig:DryLandChill}
	\end{center}
\end{figure}
\begin{figure}[H]
	\begin{center}
		\includegraphics[width=10cm]{chapters/chapter03/seqdiag/Igloo_Chill.png}
		\caption{Igloo.Chill(Tile)}
		\label{fig:IglooChill}
	\end{center}
\end{figure}
\begin{figure}[H]
	\begin{center}
		\includegraphics[width=10cm]{chapters/chapter03/seqdiag/Sea_Chill.png}
		\caption{Sea.Chill(Tile)}
		\label{fig:SeaChill}
	\end{center}
\end{figure}
\begin{figure}[H]
	\begin{center}
		\includegraphics[width=10cm]{chapters/chapter03/seqdiag/Empty_GiveTo.png}
		\caption{Empty.GiveTo(Player)}
		\label{fig:EmptyGiveTo}
	\end{center}
\end{figure}
\begin{figure}[H]
	\begin{center}
		\includegraphics[width=10cm]{chapters/chapter03/seqdiag/Food_GiveTo.png}
		\caption{Food.GiveTo(Player)}
		\label{fig:FoodGiveTo}
	\end{center}
\end{figure}
\begin{figure}[H]
	\begin{center}
		\includegraphics[width=10cm]{chapters/chapter03/seqdiag/FoodStore_Feed.png}
		\caption{FoodStore.Feed(Player)}
		\label{fig:FoodStoreFeed}
	\end{center}
\end{figure}
\begin{figure}[H]
	\begin{center}
		\includegraphics[width=10cm]{chapters/chapter03/seqdiag/ScubaGear_GiveTo.png}
		\caption{ScubaGear.GiveTo(Player)}
		\label{fig:ScubaGearGiveTo}
	\end{center}
\end{figure}
\begin{figure}[H]
	\begin{center}
		\includegraphics[width=10cm]{chapters/chapter03/seqdiag/Rope_GiveTo.png}
		\caption{Rope.GiveTo(Player)}
		\label{fig:RopeGiveTo}
	\end{center}
\end{figure}
\begin{figure}[H]
	\begin{center}
		\includegraphics[width=10cm]{chapters/chapter03/seqdiag/Part_GiveTo.png}
		\caption{Part.GiveTo(Player)}
		\label{fig:PartGiveTo}
	\end{center}
\end{figure}
\begin{figure}[H]
	\begin{center}
		\includegraphics[width=10cm]{chapters/chapter03/seqdiag/Shovel_GiveTo.png}
		\caption{Shovel.GiveTo(Player)}
		\label{fig:ShovelGiveTo}
	\end{center}
\end{figure}
\begin{figure}[H]
	\begin{center}
		\includegraphics[width=10cm]{chapters/chapter03/seqdiag/CantRescue_Rescue.png}
		\caption{CantRescue.Rescue(Tile, Tile)}
		\label{fig:CantRescueRescue}
	\end{center}
\end{figure}
\begin{figure}[H]
	\begin{center}
		\includegraphics[width=10cm]{chapters/chapter03/seqdiag/RopeRescue_Rescue.png}
		\caption{RopeRescue.Rescue(Tile, Tile)}
		\label{fig:RopeRescueRescue}
	\end{center}
\end{figure}
\begin{figure}[H]
	\begin{center}
		\includegraphics[width=10cm]{chapters/chapter03/seqdiag/PartStore_Gain.png}
		\caption{PartStore.Gain(PartStore)}
		\label{fig:PartStoreGain}
	\end{center}
\end{figure}
\begin{figure}[H]
	\begin{center}
		\includegraphics[width=10cm]{chapters/chapter03/seqdiag/BareHandsDig_Dig.png}
		\caption{BareHandsDig.Dig(Tile)}
		\label{fig:BareHandsDig.Dig}
	\end{center}
\end{figure}
\begin{figure}[H]
	\begin{center}
		\includegraphics[width=10cm]{chapters/chapter03/seqdiag/ShovelDig_Dig.png}
		\caption{ShovelDig.Dig(Tile)}
		\label{fig:ShovelDigDig}
	\end{center}
\end{figure}




%\section{State-chartok}
%\comment{Csak azokhoz az osztályokhoz, ahol van értelme. Egyetlen állapotból álló state-chartok ne szerepeljenek. A játék működését bemutató state-chart-ot készíteni tilos.}
%\newpage
%\begin{figure}[h]
%\begin{center}
%\includegraphics[width=17cm]{chapters/chapter03/example.pdf}
%\caption{x}
%\label{fig:example3}
%\end{center}
%\end{figure}

