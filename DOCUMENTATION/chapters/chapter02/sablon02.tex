% Szglab4
% ===========================================================================
%
\chapter{Követelmény, projekt, funkcionalitás}

\thispagestyle{fancy}

\section{Bevezetés}

\subsection{Cél}

%\comment{A dokumentum célja.}
Ez a dokumentum egy szoftverfejlesztési projekt információit tartalmazza az ötlettől a kész termékig, minden lépést naplózva.

\subsection{Szakterület}

%\comment{A kialakítandó szoftver milyen területen használható, milyen célra.}
A feladat játékprogram készítése, melyben a játékosok legalább háromfős csapatban működnek együtt. A program szeméyi számítógépeken, grafikus módban fog futni. A játék offline, tehát a több játékos egy számítógépen játtsza.

\subsection{Definíciók, rövidítések}
\comment{A dokumentumban használt definíciók, rövidítések magyarázata.}

\subsection{Hivatkozások}
\comment{A dokumentumban használt anyagok, web-oldalak felsorolása}

\subsection{Összefoglalás}
\comment{A dokumentum további részeinek rövid ismertetése}

\section{Áttekintés}
\subsection{Általános áttekintés}
\comment{A kialakítandó szoftver legmagasabb szintű architekturális képe. A fontosabb alrendszerek felsorolása, a közöttük kialakítandó interfészek lényege, a felhasználói kapcsolatok alapja. Esetleges hálózati és adattárolási elvárások.}

\subsection{Funkciók}
\comment{A feladat kb. 4000 karakteres (kb 1,5 oldal) részletezettségű magyar nyelvű leírása. Nem szerepelhetnek informatikai kifejezések.}

\subsection{Felhasználók}
\comment{A felhasználók jellemzői, tulajdonságai}

\subsection{Korlátozások}
\comment{Az elkészítendő szoftverre vonatkozó – általában nem funkcionális - előírások, korlátozások.}

\subsection{Feltételezések, kapcsolatok}
\comment{A dokumentumban használt anyagok, web-oldalak felsorolása}

\section{Követelmények}
\subsection{Funkcionális követelmények}


\comment{Az alábbi táblázat kitöltésével készítendő. Dolgozzon ki követelmény azonosító rendszert! Az ellenőrzés módja szokásosan bemutatás és/vagy kiértékelés. Prioritás lehet alapvető, fontos, opcionális. Az alapvető követelmények nem teljesítése végzetes. Forrás alatt a követelményt előíró anyagot, szervezetet kell érteni. Esetünkben forrás lehet maga a csapat is, mikor ő talál ki követelményt. Use-case-ek alatt az adott követelményt megvalósító használati esete(ke)t kell megadni.}

% Azonosító, Leírás, Ellenőrzés, Prioritás, Forrás, Use-case, Komment
\begin{longtable}{| l | l | l | l | l | l | l |}
\hline
\textbf{Azonosító}   & \textbf{Leírás} & \textbf{Ellenőrzés} & \textbf{Prioritás} & \textbf{Forrás} & \textbf{Use-case} & \textbf{Komment} \tabularnewline
\hline\hline
... & ... & ... & ... & ... & ... & ... \tabularnewline
\hline
\end{longtable}

\subsection{Erőforrásokkal kapcsolatos követelmények}

\comment{A szoftver fejlesztésével és használatával kapcsolatos számítógépes, hardveres, alapszoftveres és egyéb architekturális és logisztikai követelmények}

% Azonosító, Leírás, Ellenőrzés, Prioritás, Forrás, Komment
\begin{longtable}{| l | l | l | l | l | l |}
\hline
\textbf{Azonosító}   & \textbf{Leírás} & \textbf{Ellenőrzés} & \textbf{Prioritás} & \textbf{Forrás} & \textbf{Komment} \tabularnewline
\hline\hline
... & ... & ... & ... & ... & ... \tabularnewline
\hline
\end{longtable}


\subsection{Átadással kapcsolatos követelmények}
\comment{A szoftver átadásával, telepítésével, üzembe helyezésével kapcsolatos követelmények}

% Azonosító, Leírás, Ellenőrzés, Prioritás, Forrás, Komment
\begin{longtable}{| l | l | l | l | l | l |}
\hline
\textbf{Azonosító}   & \textbf{Leírás} & \textbf{Ellenőrzés} & \textbf{Prioritás} & \textbf{Forrás} & \textbf{Komment} \tabularnewline
\hline\hline
... & ... & ... & ... & ... & ... \tabularnewline
\hline
\end{longtable}

\subsection{Egyéb nem funkcionális követelmények}
\comment{A biztonsággal, hordozhatósággal, megbízhatósággal, tesztelhetőséggel, a felhasználóval kapcsolatos követelmények}

% Azonosító, Leírás, Ellenőrzés, Prioritás, Forrás, Komment
\begin{longtable}{| l | l | l | l | l | l |}
\hline
\textbf{Azonosító}   & \textbf{Leírás} & \textbf{Ellenőrzés} & \textbf{Prioritás} & \textbf{Forrás} & \textbf{Komment} \tabularnewline
\hline\hline
... & ... & ... & ... & ... & ... \tabularnewline
\hline
\end{longtable}


\section{Lényeges use-case-ek}
\comment{A 2.3.1-ben felsorolt követelmények közül az alapvető és fontos követelményekhez tartozó használati esetek megadása az alábbi táblázatos formában.}

\begin{figure}[h]
	\begin{center}
		%\includegraphics[width=17cm]{chapters/chapter02/example.pdf}
		\caption{Use-Case diagram}
		\label{fig:usecase}
	\end{center}
\end{figure}

\subsection{Use-case leírások}

\comment{Minden use-case-hez külön}

\usecase{...}{...}{...}{...}

\usecase{...}{...}{...}{...}

\section{Szótár}
\comment{A szótár a követelmények alapján készítendő fejezet. Egy szótári bejegyzés definiálásához csak más szótári bejegyzések és köznapi – a feladattól független – fogalmak használhatók fel. A szótár mérete kb. 1-2 oldal legyen.}

\section{Projekt terv}
\comment{Tartalmaznia kell a projekt végrehajtásának lépéseit, a lépések, eredmények határidejét, az egyes feladatok elvégzéséért felelős személyek nevét és beosztását, a szükséges erőforrásokat, stb. Meg kell adni a csoportmunkát támogató eszközöket, a választott technikákat! Definiálni kell, hogy hogyan történik a dokumentumok és a forráskód megosztása!}


