% Szglab4
% ===========================================================================
%
\chapter{Követelmény, projekt, funkcionalitás}

\thispagestyle{fancy}

\section{Bevezetés}

\subsection{Cél}

%\comment{A dokumentum célja.}
Ez a dokumentum egy szoftverfejlesztési projekt információit tartalmazza az ötlettől a kész termékig, minden lépést naplózva.

\subsection{Szakterület}

%\comment{A kialakítandó szoftver milyen területen használható, milyen célra.}
A feladat játékprogram készítése, melyben a játékosok legalább háromfős csapatban működnek együtt. A program szeméyi számítógépeken, grafikus módban fog futni. A játék offline, tehát a több játékos egy számítógépen játtsza.

\subsection{Definíciók, rövidítések}
%\comment{A dokumentumban használt definíciók, rövidítések magyarázata.}
Még nincs.

\subsection{Hivatkozások}
%\comment{A dokumentumban használt anyagok, web-oldalak felsorolása}
Még nincs.

\subsection{Összefoglalás}
\comment{A dokumentum további részeinek rövid ismertetése}
% TODO: összefoglalás

\section{Áttekintés}
\subsection{Általános áttekintés}
%\comment{A kialakítandó szoftver legmagasabb szintű architekturális képe. A fontosabb alrendszerek felsorolása, a közöttük kialakítandó interfészek lényege, a felhasználói kapcsolatok alapja. Esetleges hálózati és adattárolási elvárások.}
A szoftver három fő komponense a Model, a View, és a Controller. A Model reprezentálja a játék állapotát. A View kapcsolódik a Modelhez, és megjeleníti azt. A Controller felelős a felhasználói bemenetek kezeléséért és a Model frissítéséért.

\subsection{Funkciók}
%\comment{A feladat kb. 4000 karakteres (kb 1,5 oldal) részletezettségű magyar nyelvű leírása. Nem szerepelhetnek informatikai kifejezések.}
A játékban a különböző képességű szereplőknek (3 vagy több játékos lehet) kell egy tengerrel körülvett jégmezőn túlélniük. A szereplők lehetnek eszkimók vagy sarkkutatók, és körökre osztva tevékenykednek.

A jégmező jégtáblákból áll. Vannak stabil jégtáblák, amelyeken akárhány szereplő állhat, és vannak instabil jégtáblák, amik adott létszám felett átfordulnak és ilyenkor a rajtuk állók a vízbe esnek. A jégtáblákat a játék kezdetén eltérő mennyiségű hó borítja.

Az egyes jégtáblákba különféle tárgyak lehetnek belefagyva: lapát, kötél, búvárruha, élelem, stb. Befagyott tárgyat csak akkor lehet meglátni és kiásni, ha a jégtábla tiszta, nem borítja hó. A jégtáblák között lehetnek hóval fedett lukak is, amibe beleesve csak a búvárruhát viselők élnek túl, vagy azok, akiket egy köteles barátjuk a szomszéd jégtábláról azonnal kimenekít.

Minden szereplő egy körben 4 egységnyi munkát végezhet. Ilyen munka például a jégtáblán levő egységnyi mennyiségű hó eltakarítása, egy szomszédos jégtáblára való lépés vagy egy kiásott tárgy felvétele. A lapáttal két egységnyi hó takarítható el egy munkaráfordítással.

A jégmezőn időnként feltámad a hóvihar, és néhány érintett jégtáblát újabb adag friss hóval borít be. Akit elkap, annak a testhője egységnyivel csökken. Az eszkimóknak a játék elején 5 egység testhője van, a sarkkutatónak csak 4. Egy élelem eggyel növeli a testhőt.

A szereplők jégtábláról-jégtáblára haladnak képességeiknek megfelelően. A sarkkutató meg tudja nézni, hogy az a jégtábla, amire lépne, hány embert bír el (a luk egyet sem). Az eszkimó tud iglut építeni, amiben átvészelhetők a hóviharok. Egy-egy képesség alkalmazása is egy-egy munkát jelent.

A játék célja egy jelzőrakéta alkatrészeinek (pisztoly, jelzőfény, patron) megtalálása. Az alkatrészek is a jégbe vannak fagyva. Ha ezeket a csapat összegyűjti és ugyanarra a jégtáblára viszi, akkor egy munka felhasználásával összeszerelhetik és elsüthetik, amivel megnyerik a játékot. Ehhez azonban mindannyiuknak ugyanott kell állniuk. Ha valaki menet közben meghal (vízbe esve megfullad vagy elfogy a testhője és kihűl), akkor a játék véget ér. 

\subsection{Felhasználók}
\comment{A felhasználók jellemzői, tulajdonságai}
% TODO: felhasználók

\subsection{Korlátozások}
\comment{Az elkészítendő szoftverre vonatkozó – általában nem funkcionális - előírások, korlátozások.}
% TODO: korlátozások (nemfunkcionális követelmények alapján)

\subsection{Feltételezések, kapcsolatok}
%\comment{A dokumentumban használt anyagok, web-oldalak felsorolása}
Még nincs.

\section{Követelmények}
\subsection{Funkcionális követelmények}


\comment{Az alábbi táblázat kitöltésével készítendő. Dolgozzon ki követelmény azonosító rendszert! Az ellenőrzés módja szokásosan bemutatás és/vagy kiértékelés. Prioritás lehet alapvető, fontos, opcionális. Az alapvető követelmények nem teljesítése végzetes. Forrás alatt a követelményt előíró anyagot, szervezetet kell érteni. Esetünkben forrás lehet maga a csapat is, mikor ő talál ki követelményt. Use-case-ek alatt az adott követelményt megvalósító használati esete(ke)t kell megadni.}
% TODO: funkcionális követelmények

% Azonosító, Leírás, Ellenőrzés, Prioritás, Forrás, Use-case, Komment
\begin{longtable}{| l | l | l | l | l | l | l |}
\hline
\textbf{Azonosító}   & \textbf{Leírás} & \textbf{Ellenőrzés} & \textbf{Prioritás} & \textbf{Forrás} & \textbf{Use-case} & \textbf{Komment} \tabularnewline
\hline\hline
... & ... & ... & ... & ... & ... & ... \tabularnewline
\hline
\end{longtable}

\subsection{Erőforrásokkal kapcsolatos követelmények}

%\comment{A szoftver fejlesztésével és használatával kapcsolatos számítógépes, hardveres, alapszoftveres és egyéb architekturális és logisztikai követelmények}

\begin{longtable}{| l | l | p{3cm} | p{3cm} | l | l |}
\hline
\textbf{Azonosító}   & \textbf{Leírás} & \textbf{Ellenőrzés} & \textbf{Prioritás} & \textbf{Forrás} & \textbf{Komment} \tabularnewline
\hline\hline
RES\_REQ\_01 & Java futtatókörnyezet. & Az operációs rendszer csomagkezelőjében. & Elengedhetetlen. & -- & \href{https://www.java.com/en/download/}{Itt elérhető.} \tabularnewline
\hline
RES\_REQ\_02 & Java SDK. & Az operációs rendszer csomagkezelőjében. & A fordításhoz elengedhetetlen. & -- & \href{https://www.oracle.com/java/technologies/javase/javase-jdk8-downloads.html}{Itt elérhető.} \tabularnewline
\hline
\end{longtable}


\subsection{Átadással kapcsolatos követelmények}
\comment{A szoftver átadásával, telepítésével, üzembe helyezésével kapcsolatos követelmények}
% TODO: átadás követelmények

% Azonosító, Leírás, Ellenőrzés, Prioritás, Forrás, Komment
\begin{longtable}{| l | l | l | l | l | l |}
\hline
\textbf{Azonosító}   & \textbf{Leírás} & \textbf{Ellenőrzés} & \textbf{Prioritás} & \textbf{Forrás} & \textbf{Komment} \tabularnewline
\hline\hline
... & ... & ... & ... & ... & ... \tabularnewline
\hline
\end{longtable}

\subsection{Egyéb nem funkcionális követelmények}
\comment{A biztonsággal, hordozhatósággal, megbízhatósággal, tesztelhetőséggel, a felhasználóval kapcsolatos követelmények}
% TODO: nemfunkcionális követelmények

% Azonosító, Leírás, Ellenőrzés, Prioritás, Forrás, Komment
\begin{longtable}{| l | l | l | l | l | l |}
\hline
\textbf{Azonosító}   & \textbf{Leírás} & \textbf{Ellenőrzés} & \textbf{Prioritás} & \textbf{Forrás} & \textbf{Komment} \tabularnewline
\hline\hline
... & ... & ... & ... & ... & ... \tabularnewline
\hline
\end{longtable}


\section{Lényeges use-case-ek}
%\comment{A 2.3.1-ben felsorolt követelmények közül az alapvető és fontos követelményekhez tartozó használati esetek megadása az alábbi táblázatos formában.}
% TODO: use-case diagram
\begin{figure}[h]
	\begin{center}
		%\includegraphics[width=17cm]{chapters/chapter02/example.pdf}
		\caption{Use-Case diagram}
		\label{fig:usecase}
	\end{center}
\end{figure}

\subsection{Use-case leírások}

%\comment{Minden use-case-hez külön}
% TODO: forgatókönyvek
\usecase{Step}{A játékos lép.}{Player}{...}

% ez lehetne inkább Shovel
\usecase{Dig}{A játékos havat lapátol.}{Player}{...}

\usecase{Pick item up}{A játékos felvesz egy tárgyat.}{Player}{...}

\usecase{Make rocket}{A játékos jelzőrakétát készít.}{Player}{...}

\usecase{Save teammate}{A játékos kiment egy vízbe esett csapattársat.}{Player}{...}

\usecase{Build igloo}{Az eszkimó iglut épít.}{Eskimo}{...}

\usecase{Examine tile}{A sarkkutató feltérképez egy jégtáblát.}{PolarExplorer}{...}

\usecase{Turn unstable ice}{Instabil jég megfordul.}{Controller}{...}

\usecase{Create snowstorm}{Hóvihar kezdődik.}{Controller}{...}

\section{Szótár}
\comment{A szótár a követelmények alapján készítendő fejezet. Egy szótári bejegyzés definiálásához csak más szótári bejegyzések és köznapi – a feladattól független – fogalmak használhatók fel. A szótár mérete kb. 1-2 oldal legyen.}
% TODO: szótár

\section{Projekt terv}
\comment{Tartalmaznia kell a projekt végrehajtásának lépéseit, a lépések, eredmények határidejét, az egyes feladatok elvégzéséért felelős személyek nevét és beosztását, a szükséges erőforrásokat, stb. Meg kell adni a csoportmunkát támogató eszközöket, a választott technikákat! Definiálni kell, hogy hogyan történik a dokumentumok és a forráskód megosztása!}
% TODO: projekt terv


